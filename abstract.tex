%\chapter{Περίληψη}  % Add the "Abstract" page entry to the Contents
%\addtocontents{toc}{\vspace{1em}}  % Add a gap in the Contents, for aesthetics
 \abstract{


 Ο βασικός στόχος-κίνητρο της παρούσας διπλωματικής εργασίας είναι η μελέτη της καταλληλότητας του οικονομικού εγκεφαλογράφου Emotiv Epoc, για την υλοποίηση μιας διεπαφής εγκεφάλου-υπολογιστή (BCI) κάνοντας χρήση των οπτικών προκλητών δυναμικών σταθερής κατάστασης (SSVEP). Πρώτο στάδιο της διεπαφής, είναι η κατασκευή της συσκευής που παρέχει τις απαραίτητες επαναλαμβανόμενες οπτικές διεγέρσεις (ΕΟΔ). Κατασκευάζουμε ένα πάνελ αποτελούμενο από 4 LED συστοιχίες, καθώς και το κύκλωμα οδήγησης τους, το οποίο επιτρέπει την ανεξάρτητη επιλογή συχνότητας και duty cycle, για κάθε συστοιχία. Στο επόμενο στάδιο, δημιουργούμε μια εφαρμογή η οποία υλοποιεί το πρωτόκολλο καταγραφής των δεδομένων που σχεδιάσαμε, ελέγχοντας πλήρως την παραγωγή των οπτικών διεγέρσεων παρέχοντας παράλληλα στον χρήστη οπτικές και ακουστικές οδηγίες για να εξασφαλιστεί η τήρηση του πρωτοκόλλου. Επιπλέον σχεδιάζουμε μια πλήρως παραμετροποιήσιμη, γραφική διεπαφή χρήστη βασιζόμενη σε open source λογισμικό, για την real-time παρακολούθηση του εγκεφαλογραφήματος, καθώς και του συχνοτικού περιεχομένου για κάθε κανάλι. Στα SSVEP δεδομένα που συλλέξαμε, πειραματιζόμαστε με διάφορες μεθόδους εξαγωγής χαρακτηριστικών και ταξινόμησης, που χρησιμοποιούνται στην βιβλιογραφία, καθώς και δικές μας παραλλαγές τους, και επιτυγχάνουμε επιδόσεις έως 90\% ακρίβεια με ITR 20.58 bits/min, που συμβαδίζουν με τις state of the art επιδόσεις διεπαφών που κάνουν χρήση αντίστοιχου υλικού. Τέλος, υλοποιούμε το online σκέλος της διεπαφής, ένα ασύγχρονο σύστημα πραγματικού χρόνου, στο οποίο οι χρήστες καλούνται να κατευθύνουν, μέσα σε έναν εικονικό λαβύρινθο, το avatar τους, προς έναν επιθυμητό στόχο.
 \\\\\\
 \large{\textbf{Λέξεις-Κλειδιά}}\\
 Διεπαφές Εγκεφάλου - Υπολογιστή, Οπτικά Προκλητά Δυναμικά Σταθερής Κατάστασης, Canonical Correlation Analysis, Πολυμεταβλητη Γραμμική Παλινδρόμηση
 }
 \clearpage  % Abstract ended, start a new page


