%\chapter{Περίληψη}  % Add the "Abstract" page entry to the Contents
%\addtocontents{toc}{\vspace{1em}}  % Add a gap in the Contents, for aesthetics
 \abstract{


Ο βασικός στόχος-κίνητρο της παρούσας διπλωματικής εργασίας είναι η μελέτη της καταλληλότητας του οικονομικού εγκεφαλογράφου Emotiv Epoc, για την υλοποίηση μιας διεπαφής εγκεφάλου-υπολογιστή (BCI) κάνοντας χρήση των οπτικών προκλητών δυναμικών σταθερής κατάστασης (SSVEP). Πρώτο στάδιο της διεπαφής, είναι η δημιουργία της συσκευής που παρέχει τις απαραίτητες οπτικές διεγέρσεις. Κατασκευάζουμε ένα πάνελ αποτελούμενο από 4 LED συστοιχίες, καθώς και το κύκλωμα οδήγησης, το οποίο επιτρέπει την επιλογή της συχνότητας και duty cycle, για κάθε συστοιχία. Στο επόμενο στάδιο, δημιουργήσαμε μια εφαρμογή η οποία υλοποιεί το πρωτόκολλο καταγραφής των δεδομένων, ελέγχοντας πλήρως την παραγωγή των οπτικών διεγέρσεων παρέχοντας παράλληλα στον χρήστη οπτικές και ακουστικές οδηγίες για να εξασφαλιστεί η τήρηση του πρωτοκόλλου. Επιπλέον δημιουργούμε μια γραφική διεπαφή χρήστη βασιζόμενη σε open source λογισμικό και πλήρως παραμετροποιήσιμη, για την real-time παρακολούθηση του εγκεφαλογραφήματος, καθώς και του συχνοτικού περιεχομένου για κάθε κανάλι.Στα δεδομένα που συλλέξαμε, δοκιμάζουμε διάφορες μεθόδους εξαγωγής χαρακτηριστικών που χρησιμοποιούνται στην βιβλιογραφία, τα οποία χρησιμοποιούμε για να 
\textbf{τα υπολοιπα αφου τελειωσει ολο. }
 \\\\\\
 \large{\textbf{Λέξεις-Κλειδιά}}\\
 Διεπαφές Εγκεφάλου - Υπολογιστή, Οπτικά Προκλητά Δυναμικά Σταθερής Κατάστασης, Canonical Correlation Analysis, Πολυμεταβλητη Γραμμική Παλινδρόμηση
 }
 \clearpage  % Abstract ended, start a new page


