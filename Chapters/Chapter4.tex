% !TEX root = ../Thesis.tex
% !TEX output_directory
\documentclass[11pt,a4paper,english,greek,twoside]{../Thesis}

\begin{document}
\chapter{Ανασκόπηση των BCIs που κάνουν χρήση ΗΕΓ} \label{chap:survey}

\section{State of the Art χρήσεις και κατευθύνσεις}

\par Γενικά η σύγκριση μεταξύ BCI, στην πλειοψηφία των περιπτώσεων δεν είναι δυνατή. Αυτές οι διεπαφές είναι τόσο πολυμεταβλητά συστήματα, όπου είναι σχεδόν αδύνατον να αναπαραχθούν τα ίδια αποτελέσματα δεύτερη φορά, ακόμα και αν διατηρηθούν οι συνθήκες του πειράματος όσο το δυνατόν αναλλοίωτες. Συνεπώς έχοντας ως μοναδικό κριτήριο, μόνο τις επιδόσεις που αναφέρονται σε κάθε δημοσίευση, δεν μπορούμε με ασφάλεια να κάνουμε συγκρίσεις. 

\par  Μέχρι το 2000 οι υψηλότερες επιδόσεις ITR που είχαν επιτευχθεί στα BCI, κυμαίνονταν μεταξύ 5–25bits/min, \cite{wolpaw2000brain}, ενώ σήμερα αγγίζουν τα 350bits/min. Ωστόσο, αν παρατηρήσουμε την ακρίβεια που πετυχαίνει ένα τέτοιο high speed BCI, σπάνια θα ξεπερνάει το 90\%, επίδοση που πολλά από τα σύγχρονα και παλαιότερα "αργά" BCIs, υπερβαίνουν με ευκολία. Εδώ φαίνονται και οι δύο διαφορετικές προσεγγίσεις που κυριαρχούν σε αυτόν τον ερευνητικό τομέα, η μεγιστοποίηση του ITR με κάθε κόστος, και από την αντίπερα όχθη, η μεγιστοποίηση της ακρίβειας, ακόμα και αν το σύστημα είναι πιο αργό. Τα στατιστικά δείχνουν πως οι χρήστες προτιμούν BCIs με μικρότερο ITR αλλά μεγαλύτερη ακρίβεια, καθώς γίνονται πολύ λιγότερα λάθη. Επίσης, συνήθως τα BCI με πολύ υψηλές επιδόσεις όσον αφορά το ITR χρησιμοποιούν δεκάδες ΕΟΔ, πράγμα το οποίο προκαλεί κούραση στον χρήστη. Συνήθως οι ερευνητικές ομάδες που ανήκουν περισσότερο στον ιατρικό τομέα, στοχεύουν στην μεγιστοποίηση της ακρίβειας και της άνεσης του χρήστη, ενώ αυτές που ανήκουν στον πιο τεχνικό τομέα στοχεύουν στην μεγιστοποίηση του ITR. 

\par Γι αυτό το λόγο, η αναζήτηση των state of the art εργασιών, έγινε με κριτήριο την ταχύτητα του συστήματος (ITR), την ακρίβεια και την άνεση που νιώθει ο χρήστης, καθώς και την εφαρμογή πρωτοποριακών μεθόδων, ακόμα και αν δεν επιτεύχθηκαν υψηλές επιδόσεις.

\par Η εργασία η οποία ίσως έχει επηρεάσει περισσότερο από οποιαδήποτε άλλη τον ερευνητικό χώρο των SSVEP τα τελευταία χρόνια, δημοσιεύτηκε το 2007 από τους Lin et al. \cite{Lin2007-jt}, στην οποία εφαρμόστηκε για πρώτη φορά η μέθοδος CCA για την χωρική επεξεργασία των σημάτων και την εξαγωγή χαρακτηριστικών. Ήταν τέτοια η επιτυχία της μεθόδου, που έκτοτε θεωρείται ο standard τρόπος επεξεργασίας SSVEP σημάτων, και οποιαδήποτε νέα μέθοδος δημοσιευόταν, συγκρινόταν πρώτα με την CCA. Επιπλέον, αυτή η εργασία αποτέλεσε εφαλτήριο για την δημιουργία μιας σειράς αλγορίθμων βασισμένων στην CCA που στην βιβλιογραφία αναφέρονται και ως CCA-variants. Η πιο πλήρης σύγκριση όλων αυτών των μεθόδων παρουσιάστηκε το 2015 από τους Nakanishi et al. \cite{Nakanishi2015-md}, όπου συνδυάζοντας την CCA με μια παραλλαγή της, την IT-CCA, πέτυχαν ακόμα υψηλότερες επιδόσεις όπως φαίνεται στην εικόνα \ref{fig:cca_variants}. Σε πολλές άλλες δημοσιεύσεις, αυτή η συνδυαστική μέθοδος αναφέρεται ως CCA-combined, και θεωρείται ως η state-of-the-art αυτή τη στιγμή.

\begin{figure}[H]
    \centering     %%% not \center
    \includegraphics[scale=0.35]{{{ImagesSSVEP/cca_variants}.png}}
    \caption{ Διαγράμματα όπου φαίνονται τα οι επιδόσεις όλων των CCA-variants μεθόδων, καθώς και η ανωτερότητα της CCA-combined (Combination Method). Εικόνα από \cite{Nakanishi2015-md}.
    }
    \label{fig:cca_variants}
\end{figure}

\par Την ίδια χρονιά που παρουσιάστηκε η CCA-combined, η ίδια ερευνητική ομάδα κατέγραψε την υψηλότερη ITR επίδοση μέχρι τότε, υλοποιώντας έναν speller, αγγίζοντας τα 320bits/min (5.33bits/s) \cite{Chen2015-oh}, ενώ το 2018, αναζητώντας βελτιώσεις στο προηγούμενο σύστημα τους, πέτυχαν ακρίβεια 89.83±6.07\% και ITR ίσο με 325.33±38.17 bits/min για σύγχρονη (synchronous, cue-based) online διεπαφή \cite{Nakanishi2018-sc}. Αυτή η επίδοση αποτελεί σήμερα, την υψηλότερη αναφερθείσα στην βιβλιογραφία, ωστόσο αναφέρουν πως στην ασύγχρονη εκδοχή του συστήματός τους, η μέση επίδοση των χρηστών έπεσε στα 198.67±50.48 bits/min. Η διεπαφή που ανέπτυξαν είναι ένας speller όπου τα γράμματα και οι αριθμοί είναι διατεταγμένα σε έναν πίνακα 8x5, και καθένα από τα 40 κελιά, είναι μια ΕΟΔ διαφορετικής συχνότητας. Ωστόσο μια σημαντική διαφορά με άλλες εργασίες είναι πως κάθε ΕΟΔ ταλαντώνεται με διαφορετική φάση, η οποία με την σειρά της λαμβάνεται υπόψιν κατά την επεξεργασία των σημάτων. 

\begin{figure}[H]
    \centering     %%% not \center
    \includegraphics[scale=0.4]{{{ImagesSSVEP/survey}.png}}
    \caption{ Διάγραμμα με τις καλύτερες επιδόσεις όσον αφορά τον ITR, μέχρι το 2015. Η υψηλότερη επίδοση μέχρι τότε άγγιζε τα 320bits/min (5.33bits/s) \cite{Chen2015-oh}. Εικόνα από \cite{Chen2015-oh}.
    }
    \label{fig:survey}
\end{figure}

%\cite{Lawhern2018-to} EEG-net
 Μία από τις πιο πρόσφατη έρευνα που βρήκαμε πάνω στα SSVEP \cite{Waytowich_undated-iu}, κάνει χρήση ενός νευρωνικού δικτύου, και αναφέρει πως τα αποτελέσματα ήταν βελτιωμένα σε σχέση με τον CCA-combined. Επιπλέον αναφέρουν, πως ενώ δεν ήταν αυτή η αρχική τους πρόθεση, τα features που "μάθαινε" το νευρωνικό είχαν σχέση με και την φάση κάθε πηγής ΕΟΔ, και ότι αυτός ήταν ένας από τους λόγους της επιτυχίας. Γενικά, κάνοντας μια αναζήτηση των εργασιών που προσεγγίζουν το ζήτημα, κυρίως με την χρήση νευρωνικών δικτύων, ο αριθμός τους δεν ξεπέρναγε τις πέντε, γεγονός το οποίο υποδεικνύει μια πιθανή μελλοντική κατεύθυνση των ερευνών. 

Οι Dreyer et al. το 2015 \cite{Dreyer2015-ie} και το 2017 \cite{Dreyer2017-pq}, προσπαθώντας να μειώσουν την κούραση που βιώνει ο χρήστης παρατηρώντας τις ΕΟΔ,  δοκίμασαν την χρήση ΕΟΔ, των οποίων η συχνότητα είχε διαμορφωθεί με ένα υψίσυχνο φέρον (FM-modulation). Η επίδοση του συστήματος ήταν ελαφρώς χαμηλότερη από την κλασσική μέθοδο, ωστόσο οι χρήστες ανέφεραν σημαντικά μειωμένη κόπωση κοιτώντας τις διαμορφωμένες ΕΟΔ. Επιπλέον, αναφέρουν πως επικεντρώθηκαν στην κυρίως δημιουργία των ΕΟΔ και της γενικής υλοποίησης, χωρίς να προσπαθήσουν να βελτιστοποιήσουν τον αλγόριθμο για την παραγωγή καλύτερων αποτελεσμάτων. Συνεπώς καταλήγουν, η αντοχή των χρηστών κατά την διαδικασία είναι προτεραιότητα, και πως η μελλοντικές έρευνες πρέπει να στραφούν προς τις FM-ΕΟΔ, για την περαιτέρω εξέλιξη τους.
 

Μιά ακόμα, προτότυπη ερευνητική εργασία, δημοσιεύτηκε το 2017,\cite{Maye2017-ny}, στην οποία μελετήθηκε η επίδραση των σχετικών θέσεων των ΕΟΔ, ως προς τον χρήστη, και το κατά πόσο αυτή η θέση επηρεάζει την τοπολογία των SSVEP σημάτων στον εγκέφαλο. Συγκεκριμένα, χρησιμοποιήθηκαν τέσσερις ΕΟΔ, ίδιας ακριβώς συχνότητας και φάσης, των οποίων η θέση σχημάτιζε έναν νοητό κυκλικό δίσκο. Αυτό που παρατηρήθηκε, είναι πως αναλόγως της ΕΟΔ που κοιτούσε ο χρήστης, ο κύριος όγκος των SSVEP σημάτων, εμφανιζόταν σε ελαφρώς διαφορετικές τοποθεσίες στον εγκέφαλο, καθιστώντας ικανό τον διαχωρισμό τους, καθώς η επίδοση του συστήματος ήταν 95 ± 3\% πιστότητα, με ITR 40.8 ± 3.3 bits/min. Αυτή η μέθοδος έχει δύο βασικά πλεονεκτήματα. Αρχικά λόγω της παρουσίας μόνο μίας συχνότητας διέγερσης, υπάρχουν λιγότερα συχνοτικά "παράσιτα" στο φάσμα των SSVEP, συνεπώς είναι πιο εύκολη η ανίχνευση τους. Κατά δεύτερον, η διεπαφή γίνεται πολύ πιο ξεκούραστη για τον χρήστη, καθώς το κοίταγμα σε πολλές ΕΟΔ με διαφορετικές συχνότητες μπορεί να προκαλέσει ζάλη. Μια τελευταία παρατήρηση είναι πως σε πολλές εργασίες, γίνεται η προσπάθεια εύρεσης των κατάλληλων συχνοτήτων για κάθε χρήστη, όμως η πληθώρα όλων των δυνατών συνδυασμών, καθιστά απαγορευτική την εύρεση του βέλτιστου συνδυασμού. Με αυτή την μέθοδο πλέον, ο χώρος αναζήτησης συχνοτήτων, μειώνεται δραματικά, ωστόσο ίσως να απαιτείται περαιτέρω έρευνα για την βέλτιστη τοποθεσία των ΕΟΔ. 

\par Τέλος, σε μια προσπάθεια αύξησης τον επιδόσεων των BCI, μια νέα τάση αποτελούν τα υβριδικά BCI \cite{Hong2017-kl}, τα οποία κάνουν συνδυασμό διαφόρων μεθόδων, όχι απαραίτητα σχετικών με EEG σήματα. Για παράδειγμα ο συνδυασμός Motor Imagery - SSVEP για τον έλεγχο αναπηρικού αμαξιδίου \cite{Cao2014-wz}, SSVEP-P300 \cite{Chang2016-ax}, και EEG-eye tracking \cite{Kim2014-gg} \cite{Stawicki2017-ia}. Ειδικά η χρήση των eye trackers σε συνδυασμό με το EEG, φαίνεται να είναι ένας πρόσφορος τομέας για έρευνα, και προς αυτή την κατεύθυνση, από το 2017, η Guger Technologies (g.tec), μια από τις πιο γνωστές κατασκευάστριες EEG συστημάτων, δίνει την δυνατότητα αγοράς των εγκεφαλογράφων της μαζί με ένα ζευγάρι ειδικά γυαλιά που έχουν ενσωματωμένο σύστημα eye traciking.


\section{Παρόμοιες εργασίες - Διεπαφές βασισμένες σε Emotiv Epoc και SSVEPs}

\par Σε αυτό το σημείο θα κάνουμε μια σύντομη παρουσίαση των δημοσιεύσεων που αναφέρονται περισσότερο στη βιβλιογραφία. Η αναζήτηση έγινε χρησιμοποιώντας ως λέξεις κλειδιά τις "Emotiv", "Epoc", "SSVEP". Ο σκοπός είναι ενημερωθούμε για τις επιδόσεις διεπαφών που χρησιμοποίησαν παρόμοιο υλικό με εμάς, και να αποτελέσουν ένα σημείο αναφοράς και σύγκρισης των δικών μας αποτελεσμάτων.

\par Η πιο αναφερόμενη (cited) έρευνα πάνω στο θέμα Epoc - SSVEP, δημοσιεύτηκε το 2012 \cite{Liu2012-qj} και ο σκοπός ήταν η σύγκριση του Epoc με ένα σύστημα ιατρικών προδιαγραφών, το  g.USBamp. H μέθοδος που χρησιμοποίησαν ήταν η CCA, πραγματοποίησαν offline και online ανάλυση σε τέσσερις χρήστες. Στην οffline χρησιμοποιήθηκαν 16 ΕΟΔ και χρονικά παράθυρα 6sec, ενώ στην online 6 ΕΟΔ. Τα αποτελέσματα της offline ανάλυσης για το EPOC ήταν 82.99±4.98\% ακρίβεια με ITR 28.06±6.45 bits/min. Στο online σκέλος, οι επιδόσεις είναι 95.83±3.59\%, με ITR 18.99±1.68 bits/min και χρόνο αναγνώρισης για κάθε εντολή 5.25±2.14 sec. Ωστόσο δεν διασαφηνίζουν αν αυτά τα αποτελέσματα αφορούν το Epoc, η τον g.USBamp, καθώς επίσης δεν αναφέρουν αν η online διεπαφή ήταν σύγχρονη, έτσι ώστε να μπορούν να υπολογίσουν τον ITR. 

\par Σε μια άλλη δημοσίευση \cite{noauthor_undated-vk}, πάλι ο Epoc συγκρίθηκε με έναν ακριβότερο εγκεφαλογράφο, αλλά σε τελείως διαφορετικές καταστάσεις από την προηγούμενη εργασία. Ο σκοπός ήταν ο έλεγχος ενός παιχνιδιού, κάνοντας χρήση μόνο μιας ΕΟΔ, και χρησιμοποιώντας ως μέθοδο ανίχνευσης SSVEP, κατάτμηση του σήματος στο πεδίο του χρόνου και averaging \cite{Friman2007-xm}. Παρότι το πλαίσιο εφαρμογής αυτής της δημοσίευσης διαφέρει σημαντικά από την παρούσα εργασία, είναι σημαντικό να αναφερθεί το γεγονός πως αυτή η διεπαφή δεν έμεινε μόνο στο εργαστήριο, αλλά δοκιμάστηκε από 25 άτομα, σε εξωτερικό χώρο, όπου 9 από αυτά την χειρίστηκαν με άνεση.

% Φαίνεται πως το Epoc αποδίδει καλά στα σήματα P300 αν και δεν έχει ηλεκτρόδια εκει που πρεπει Debener S, Minow F, Emkes R, Gandras K, de Vos M: How about taking a low-cost, small, and wireless EEG for a walk?. Psychophysiology. 2012, 49: 1617-1621.
% [[
% [HTML] Performance of the Emotiv Epoc headset for P300-based applications
% M Duvinage, T

%D. Matthieu, C. Thierry, P. Mathieu, A P300-based quantitative comparison between the Emotiv EPOC headset and a medical EEG device. Int. J. Biomed. Eng. 12(56), 201 (2013)

\par Μια άλλη πολύ σημαντική έρευνα, έγινε το 2014 στο πανεπιστήμιο UCSD \cite{Lin2014-cp}, και ανοίγει τον δρόμο για την χρήση των BCI εκτός εργαστηρίου, σε κανονικές συνθήκες. Συγκεκριμένα, κάνοντας χρήση του Epoc, δοκιμάστηκε το κατά πόσο είναι δυνατή η υλοποίηση μια SSVEP διεπαφής, την ώρα που ο χρήστης περπατάει με διάφορες ταχύτητες. Στα αποτελέσματά τους παρουσίασαν πως για ταχύτητες έως και 0.89m/s, μπορεί να επιτευχθεί ITR έως και 12bits/min. Επιπλέον, παρουσιάζονται και οι επιδόσεις για την περίπτωση όπου ο χρήστης είναι ακίνητος, μέση ακρίβεια 76.60 ± 21.74\% με ITR: 14.38 ± 9.04 για 17 χρήστες.
Το γεγονός σε αυτή την δημοσίευση χρησιμοποιήθηκαν τέσσερις ΕΟΔ με συχνότητες 9,10,11 και 12 Hz, και φυσικά ο Epoc, την καθιστούν αρκετά παρόμοια με την παρούσα διπλωματική, και συνεπώς θα έχει αξία να δούμε αν τα αποτελέσματα μας θα είναι συγκρίσιμα.
% στο ιδιο πεηπερ λεει :The amplitude of SSVEP has been found to be largely modulated by visual spatial attention [28]. As a consequence, the loss of focus could reduce visual attention and thereby lead to the decreased SSVEP amplitude.
% Επισης εδω καταληγουν πως ο Εποκ εχει διαφορα προβληματακια, και δεν πρεπει να χρησιμοποιείται σε σοβαρες εφαρμογες (να το βαλω στο τελος στα συμπερασματα)

\begin{figure}[H]
    \centering     %%% not \center
    \includegraphics[scale=0.5]{{{ImagesSSVEP/walking_SSVEP}.png}}
    \caption{ Η πειραματική διάταξη της εργασίας \cite{Lin2014-cp}, όπου έδειξαν πως είναι δυνατή η υλοποίηση διεπαφών σχετικά υψηλών επιδόσεων, για χρήση σε πραγματικές συνθήκες έξω από το εργαστήριο.}
    \label{fig:keep_walking}
\end{figure}

\par Η μόνη εργασία που βρήκαμε, η οποία να χρησιμοποιεί ΕΟΔ με υψηλές συχνότητες 28, 30, 32 και 34 Hz, είναι η \cite{Holewa2014-an}, και τα αποτελέσματα 73.75\% ακρίβεια, με ITR 11.36, τα οποία αν και υπολείπονται με τα προηγούμενα, είναι πολύ ικανοποιητικά δεδομένου του χαμηλού SNR που παρουσιάζουν τα SSVEP στις συχνότητες αυτές. %\textbf{δεν μου το ανοιγει το scihub για να δω λεπτομερειες}
% παλι δν περιγραφουν τπτ απο την μεθοδο τους. 


%http://www.koreascience.or.kr/article/ArticleFullRecord.jsp?cn=PJJNBT_2015_v25n3_254
%εδω πετυχαινουν 70\% με LDA kai SVM, αλλα το κειμενο ειναι στα κορεατικα

\par Τέλος, να αναφερθούμε και σε μια δημοσίευση που ισχυρίζεται πως το Epoc δεν είναι κατάλληλο για την ανίχνευση SSVEPs \cite{noauthor_undated-hj}. Ο αρχικός σκοπός της εργασίας ήταν η χρήση νευρωνικού δικτύου για την κατηγοριοποίηση εγκεφαλικών σημάτων που παράγει ο χρήστης από μόνος του, χωρίς την επίδραση εξωτερικών διεγέρσεων (active BCI). Ωστόσο, αναζητώντας διαφορετικούς τρόπους υλοποίησης, στράφηκαν προς τα SSVEPs, δοκιμάζοντας να ανιχνεύσουν τα δυναμικά χρησιμοποιώντας μια ΕΟΔ συχνότητας 7Hz. Τελικώς, δεν κατάφεραν να εντοπίσουν SSVEP σήματα, ισχυριζόμενοι πως ο βασικός παράγοντας ήταν η έλλειψη ηλεκτροδίων στον ινιακό λοβό, καθώς και ότι τα η θέση των ηλεκτροδίων O1 O2, δεν είναι κατάλληλα ρυθμισμένη στον Epoc .

\end{document}