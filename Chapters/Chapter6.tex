% !TEX root = ../Thesis.tex
% !TEX output_directory
\documentclass[11pt,a4paper,english,greek,twoside]{../Thesis}

\begin{document}
\chapter{Αποτελέσματα - Συμπεράσματα} \label{chap:Results}

\section{Αποτελέσματα Offline ανάλυσης}

\subsection{Μεθοδολογία Εξαγωγής Αποτελεσμάτων}

\par Όπως αναφέραμε και στο κεφάλαιο 5ο, 2 άτομα συμμετείχαν στην καταγραφή των δεδομένων, και καθένα από αυτά πήρε μέρος σε 4 διαφορετικές συνεδρίες. Ο λόγος που χρειάστηκαν τόσες, ήταν για να δοκιμάσουμε διαφορετικές παραμέτρους όσον αφορά την ΕΟΔ. Συγκεκριμένα, πειραματιστήκαμε με δύο διαφορετικές τιμές duty cycle (50\%, 75\%) καθώς και δύο διαφορετικές ομάδες συχνοτήτωv. Η πρώτη ομάδα είναι $F_{low} = [6Hz,7Hz,8Hz,9Hz]$, και η δεύτερη $F_{mid} = [12Hz,13Hz,14Hz,15Hz]$.

\par Τα είδη των αλγορίθμων με τους οποίους θα πειραματιστούμε είναι δύο. Αυτοί που δεν χρειάζονται δεδομένα εκπαίδευσης (PSD, CCA) και αυτοί που χρειάζονται (PCA, MLR, kNN). Για την αξιολόγηση των αλγορίθμων της δεύτερης κατηγορίας, χρησιμοποιήσαμε leave-one-out cross validation (loocv), δηλαδή για κάθε δείγμα κρατούσαμε αυτό ως δείγμα δοκιμής (test data) και το ταξινομούσαμε χρησιμοποιώντας το εκάστοτε μοντέλο που εκπαιδεύαμε με βάση τα εναπομείναντα δείγματα (train data).

\par Μια άλλη παράμετρος για την οποία θα παρουσιάσουμε αποτελέσματα είναι το μέγεθος του κάθρ epoch στο οποίο εφαρμόζεται ο εκάστοτε αλγόριθμος. Επιπλέον πέρα από τα SSVEP σήματα, θα δοκιμάσουμε τις επιδόσεις των αλγορίθμων και στην ανίχνευση alpha κυμάτων, τα οποία, όπως αναφέραμε, μπορούμε να τα θεωρήσουμε ως SSVEP προκαλούμενα από ΕΟΔ συχνότητας ίδιας με αυτή των alpha.

\par Τέλος θα σχολιάσουμε την ικανότητα του συστήματος μας να ανιχνεύει την NC, τον ρυθμό των false positives, και θα επιλέξουμε και τον αλγόριθμο ο οποίος ανιχνεύει καλύτερα τα άλφα κύματα. 

\subsection{Συνολικά αποτελέσματα για χρονικό παράθυρο $t=5sec$}

\par Έπειτα από σχολαστική ανασκόπηση της βιβιογραφίας, παρατηρήσαμε πως σχεδόν πάντα το ποσοστό επιτυχίας μιας μεθόδους, βελτιώνεται όσο αυξάνεται το παράθυρο υπολογισμού (διάρκεια epoch). Γι' αυτό το λόγο, θα παρουσιάσουμε τα συνολικά αποτελέσματα και για τις 4 συνεδρίες κάθε ατόμου, για $t=5sec$, έτσι ώστε να επιλέξουμε αρχικά τις βέλτιστες συχνότητες και τιμές duty cycle, για τον καθένα, όντας σχεδόν σίγουροι πως η επιλογή μας δεν θα άλλαζε αν ελέγχαμε μικρότερα χρονικά παράθυρα. Στην συνέχεια αφού γίνουν αυτές οι επιλογές, θα παρουσιαστούν αναλυτικά τα αποτελέσματα ταξινόμησης για κάθε μέθοδο, και για κάθε δυνατή διάρκεια χρονικού παραθύρου.

\begin{table}[H]
    \centering
    \begin{tabular}{ |p{4cm}||p{1cm}|p{1cm}|p{1cm}|p{1cm}|p{1cm}|p{1cm}|p{1cm}|p{1cm}|}
        \hline
        \multicolumn{1}{|c||}{}& \multicolumn{8}{c|}{Χρήστης S1}\\
        \hline
        & \multicolumn{4}{c|}{$F_{low}$} & \multicolumn{4}{c|}{$F_{mid}$} \\
        \hline
        & \multicolumn{2}{c|}{$50\%$} & \multicolumn{2}{c|}{$75\%$} &
        \multicolumn{2}{c|}{$50\%$} & \multicolumn{2}{c|}{$75\%$} \\
        \hline
        & \multicolumn{1}{c|}{$A_c$} & \multicolumn{1}{c|}{$ITR$} &
         \multicolumn{1}{c|}{$A_c$} & \multicolumn{1}{c|}{$ITR$} &
          \multicolumn{1}{c|}{$A_c$} & \multicolumn{1}{c|}{$ITR$} &
           \multicolumn{1}{c|}{$A_c$} & \multicolumn{1}{c|}{$ITR$} \\
        \hline
        PSD             & 0.5375& 3.25& 0.575& 4.11& 0.6125& 5.07& 0.5875& 4.42 \\
        PSD-GM          & 0.85& 13.82& 0.875& 15.09& 0.825& 12.64& 0.7875&  11.00 \\
        CCA             & 0.90& 16.47& 0.90& 16.47& 0.85& 13.82& 0.7625& 9.99 \\
        CCA-kNN         & 0.925& 17.96& 0.9125& 17.19& 0.8625& 14.45& 0.8125&  12.07 \\
        PSD-PCA-MLR-kNN & 0.9625& 20.51& 0.975& 21.50& 0.9375& 18.76& 0.875& 15.09  \\
        \hline
        \multicolumn{5}{c}{}\\
        \hline
        \multicolumn{1}{|c||}{}& \multicolumn{8}{c|}{Χρήστης S2}\\
        \hline
        & \multicolumn{4}{c|}{$F_{low}$} & \multicolumn{4}{c|}{$F_{mid}$} \\
        \hline
        & \multicolumn{2}{c|}{$50\%$} & \multicolumn{2}{c|}{$75\%$} &
        \multicolumn{2}{c|}{$50\%$} & \multicolumn{2}{c|}{$75\%$} \\
        \hline
        & \multicolumn{1}{c|}{$A_c$} & \multicolumn{1}{c|}{$ITR$} &
         \multicolumn{1}{c|}{$A_c$} & \multicolumn{1}{c|}{$ITR$} &
          \multicolumn{1}{c|}{$A_c$} & \multicolumn{1}{c|}{$ITR$} &
           \multicolumn{1}{c|}{$A_c$} & \multicolumn{1}{c|}{$ITR$} \\
        \hline
        PSD             & 0.50& 2.49& 0.5625& 3.81& 0.5125& 2.73& 0.475& 2.03 \\
        PSD-GM          & 0.7875& 11.00& 0.8125& 12.07& 0.7625& 9.99& 0.65& 6.13 \\
        CCA             & 0.8875& 15.77& 0.8375& 13.22& 0.8375& 13.22& 0.75& 9.50\\
        CCA-kNN         & 0.8625& 14.45& 0.85& 13.82& 0.8625& 14.45& 0.7375& 9.04 \\
        PSD-PCA-MLR-kNN & 0.8875& 15.77& 0.9& 16.47& 0.85& 13.82& 0.8 & 11.53 \\
        \hline
    \end{tabular}
    \caption{Συγκεντρωτικά αποτελέσματα ακρίβειας $A_c$ και ρυθμού $ITR (bits/min)$ κάθε χρήστη, για κάθε συνδυασμό συχνοτήτων ($F_{low}, F_{mid}$, duty cycle ($50\%, 75\%$), και μεθόδων που δοκιμάστηκαν, για χρονικό παράθυρο t=5sec}
    \label{tab:full_results}
\end{table}

\par Στον πίνακα \ref{tab:full_results}, παρουσιάζουμε τα αποτελέσματα για το πρόβλημα της ταξινόμησης των SSVEP στις τέσσερις κλάσεις, χωρίς να συμπεριλαμβάνουμε τις καταστάσεις NC και "κλειστά μάτια" (alpha κύματα). Ο χρήστης S1 είχε περισσότερη εμπειρία στην χρήση διεπαφών SSVEP, ενώ ο χρήστης S2 δοκίμαζε για πρώτη φορά. Ίσως αυτός είναι και ο λόγος που ο S1 πέτυχε τα καλύτερα αποτελέσματα, σχεδόν σε όλες τις περιπτώσεις. Παρά το ότι γενικά δεν απαιτείται εκπαίδευση για την παραγωγή των SSVEPs, είναι γεγονός, πως ένας πιο έμπειρος χρήστης μπορεί να διαχειριστεί καλύτερα την κούραση που ίσως προκύψει κοιτώντας τις ΕΟΔ, καθώς επίσης έχει καλύτερο έλεγχο και συγκέντρωση κατά την διάρκεια της καταγραφής.
\par Όσον αφορά τώρα τις δύο διαφορετικές παραμέτρους ΕΟΔ που δοκιμάσαμε (duty cycle, συχνότητες) έχουμε να πούμε τα εξής. Είναι γεγονός, πως μεταξύ καθεμίας από τις τέσσερις συνεδρίες παρεμβάλλονταν αρκετές ώρες οι και μέρες, δηλαδή η κατάσταση του κάθε χρήστη διέφερε αρκετά. Συνεπώς, για να είμαστε απόλυτα σίγουροι πως αυτές οι αλλαγές που παρατηρούμε στις επιδόσεις, ευθύνονται στις διαφορετικές επιλογές duty cycle και συχνοτήτων, θα έπρεπε οι συνεδρίες να γίνουν τουλάχιστον την ίδια μέρα και διαδοχικά. Ωστόσο, παρατηρούμε μια συνέπεια και στους δύο χρήστες, όσον αφορά το γεγονός πως η χρήση χαμηλότερου duty cycle ευνοεί τις υψηλότερες συχνότητες, και αντίστροφα. Η παρατήρηση αυτή συμβαδίζει με τα αποτελέσματα που παρουσιάστηκαν στην δημοσίευση \cite{Huang2012-va} όπου φάνηκε πως το πλάτος των SSVEP σημάτων που αντιστοιχούσαν σε ΕΟΔ με συχνότητα 13Hz, ήταν μέγιστο για duty cycle 30\%, ενώ για συχνότητα 7Hz, το μέγιστο πλάτος παράχθηκε όταν χρησιμοποιήθηκε duty cycle 75\%.

\section{Χρήστης S1}
\par Σε αυτό το σημείο θα παραθέσουμε πιο λεπτομερή αποτελέσματα για τον χρήστη S1, και συγκεκριμένα για την καταγραφή όπου χρησιμοποιήθηκαν οι ΕΟΔ $F_{low}$ και duty cycle 75%.
\par Tableράσματα for SSVEP, diagram for SSVEP, table for alpha, diagram for alpha, table with zero, diagram with zero
\par For the best method show confusion matrix, and details, kai ITR
\subsection{Χρήστης S2}
\par Table for SSVEP, diagram for SSVEP, table for alpha, diagram for alpha, table with zero, diagram with zero
\par For the best method show confusion matrix, and details, kai ITR


\section{Αξιολόγησης Online διεπαφής}

\par Όπως αναφέρθηκε και στο κεφάλαιο 2 η 5, η χρήση της μετρικής ITR δεν αποτελεί έγκυρο τρόπο αξιολόγησης ενός ασύγχρονου συστήματος πραγματικού χρόνου όπως αυτό που υλοποιήσαμε. Αντιθέτως, σύμφωνα με την δημοσίευση, \cite{Yuan2013-jp}, ως μέτρο επίδοσης της διεπαφής μας, ορίσαμε τον χρόνο που απαιτείται για την ολοκλήρωση μιας συγκεκριμένης διαδικασίας από τον χρήστη, όπως η εύρεση του σωστού μονοπατιού σε έναν λαβύρινθο.

\par Και οι 2 χρήστες κλήθηκαν να εκτελέσουν την ίδια ακριβώς διαδικασία. Υπολογίστηκε πως ο ελάχιστος χρόνος ολοκλήρωσης της διαδικασίας, αν το ποσοστό αναγνώρισης είναι 100\%, είναι περίπου 40 με 45 δευτερόλεπτα. Οι χρόνοι που επιτεύχθηκαν στην πραγματικότητα είναι αυτοί που φαίνονται στον πίνακα \ref{tab:online}:

\begin{table}[H]
    \centering
    \begin{tabular}{ |p{1cm}||p{1cm}|p{1cm}|p{1cm}|p{1cm}|p{1cm}|p{1cm}|}
        \hline
        & \multicolumn{3}{c|}{Asynchronous} & \multicolumn{3}{c|}{Constant-Engaged} \\
        \hline
        & \multicolumn{1}{c|}{Time (sec)} & \multicolumn{1}{c|}{Rating} &
        \multicolumn{1}{c|}{Alpha Commands} & \multicolumn{1}{c|}{Time (sec)} & \multicolumn{1}{c|}{Rating} &
        \multicolumn{1}{c|}{Alpha Commands} \\
        \hline
        S1          & 125& 4& 4& 100& 3& 2 \\
        S2          & 125& 4& 4& 100& 3& 2 \\
        \hline
    \end{tabular}
    \caption{Συγκεντρωτικά αποτελέσματα ακρίβειας $A_c$ και ρυθμού $ITR (bits/min)$ κάθε χρήστη, για κάθε συνδυασμό συχνοτήτων ($F_{low}, F_{mid}$, duty cycle ($50\%, 75\%$), και μεθόδων που δοκιμάστηκαν, για χρονικό παράθυρο t=5sec}
    \label{tab:online}
\end{table}

\par Η επίδοση των χρηστών κατά τις δοκιμές του online συστήματος μπορεί να είναι σημαντικά μικρότερη από αυτήν των offline δοκιμών, και υπάρχουν αρκετοί λόγοι γι' αυτό, όπως η διάσπαση της προσοχής του χρήστη λόγω της ανάδρασης του συστήματος (οθόνη, ενδείξεις κ.α)\cite{Muller-Putz2006-wj} \cite{Yuan2013-jp}.

\section{Γενικά Συμπεράσματα}
1) Αρχικά ο σκοπός της διπλωματική εργασίας ήταν να διαπιστώσουμε το κατά πόσο το EPOC είναι ικανό να ανιχνεύσει SSVEP σήματα. Παρά την έλλειψή αρκετών ηλεκτροδίων στην ινιακή περιοχή του εγκεφάλου, όπου εμφανίζονται τα SSVEP, τα αποτελέσματα ήταν αρκετά ικανοποιητικά, και σε κάποιες περιπτώσεις ...

2) Το γεγονός πως ο ΕΠΟΚ δεν επιτρέπει την ακριβή τοποθέτηση των ηλεκτρόδιων, προκαλεί δυσκολίες στην επεξεργασία των σημάτων Συγκεκριμένα στις μεθόδους μηχανικής μάθησης, που απαιτείται πρώτα ένα training session, προκειμένου να δημιουργηθούν οι πίνακες μετασχηματισμού για κάθε χρήστη, παρατηρήσαμε πως δεν έδωσαν ικανοποιητικά αποτελέσματα στο online σκέλος της διεπαφής, πράγμα το οποίο δεν θα έπρεπε να μας εκπλήσσει, καθώς τα μοντέλα "έμαθαν" να εξηγούν εγκεφαλικά σήματα που λήφθηκαν κάτω από συγκεκριμένες συνθήκες, ενώ κατά την διάρκεια των real-time δοκιμών, κλήθηκαν να "αποφασίσουν" για εγκεφαλικά σήματα που λήφθηκαν κάτω από διαφορετικές συνθήκες, και ελαφρώς διαφορετική διάταξη ηλεκτροδίων. 

\par Ένα άλλο γεγονός που αποτελεί ένδειξή της σημαντικότητας της τοποθεσίας των ηλεκτροδίων, είναι το γεγονός πως η μέθοδος CCA έδωσε καλά αποτελέσματα τόσο στην offline όσο και στην online περίπτωση. Πέραν του γεγονότος πως η μέθοδος CCA δεν απαιτούσε δεδομένα εκπαίδευσης, έχει άλλο ένα ιδιαίτερο χαρακτηριστικό. Από πολλούς θεωρείται και χρησιμοποιείται ως τεχνική χωρικού φιλτραρίσματος \cite{Spuler2014-wl}\cite{Clercq2006-jn}. Πράγματι μπορούμε να θεωρήσουμε τα $a,b$ της εξίσωσης \textbf{3.8} ως τα χωρικά φίλτρα τα οποία αυξάνουν τον SNR. Ως αποτέλεσμα, αυτή η μέθοδος είναι πιο ανεκτική σε μικρές αλλαγές της τοποθεσίας των ηλεκτροδίων μεταξύ των διαφόρων καταγραφών. 


\end{document}