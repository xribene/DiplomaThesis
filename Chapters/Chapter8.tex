%% !TEX root = ../Thesis.tex
%% !TEX output_directory
\documentclass[11pt,a4paper,english,greek,twoside]{../Thesis}




\begin{document}
\chapter{Συμπεράσματα-Επίλογος}
\section{Συμβολή της Διπλωματικής Εργασίας}
Στην εργασία αυτή αντιμετωπίσαμε τα ζητήματα αναγνώρισης και κατάτμησης δράσεων λεπτομέρειας σε βίντεο με χρήση πολλών καναλιών πληροφορίας. Δώσαμε βάρος στη σημαντικότητα της σημασιολογίας στην αναγνώριση δράσεων και δείξαμε ότι η σύζευξη αυτής με ισχυρά χαρακτηριστικά χαμηλού επιπέδου Όρασης Υπολογιστών μπορεί να βελτιώσει αισθητά το αποτέλεσμα ταξινόμησης σε σχέση με τη χρήση αποκλειστικά οπτικών χαρακτηριστικών.

\par Η εργασία μας εκκίνησε με την παρουσίαση ενός γενικευμένου, ενιαίου συστήματος κατάτμησης και αναγνώρισης δράσεων το οποίο συνδυάζει πολλαπλά κανάλια πληροφορίας όπως Όραση, Κείμενο και Ομιλία (μέσω υποτίτλων). Ο συνδυασμός των διαφορετικών καναλιών φαίνεται να συνεισφέρει τόσο στην αναπαράσταση των δεδομένων, όσο και στην αποτελεσματικότερη αναγνώριση και τελικά κατάτμηση. Μαζί με τη γενικευμένη σχεδίαση, παρουσιάσαμε και συγκεκριμένη υλοποίηση του προτεινόμενου συστήματος, εξηγώντας τις παραμέτρους που επιλέξαμε και εφαρμόζοντάς το σε ένα απαιτητικό σύνολο δεδομένων βίντεο \cite{rohrbach_2012}.

\par Καθώς το σύνολο δεδομένων δεν παρείχε το σύνολο των επισημειώσεων που θα θέλαμε να χρησιμοποιήσουμε, προβήκαμε στη δημιουργία δικών μας επισημειώσεων. Πιο αναλυτικά, με τη βοήθεια επισημειωτών, συλλέχθηκαν αρχεία υποτίτλων για τα βίντεο. Ακόμη, προβήκαμε σε επισημειώσεις αντικειμένων ανά 100 frames βίντεο, κοιτώντας για αντικείμενα που εξασφαλίζουν την ελάχιστη οπτική διασπορά στη διάρκεια του βίντεο. Επιπλέον, χρησιμοποιήσαμε μια νεότερη μορφή του συνόλου δεδομένων \cite{rohrbach_2015} για να συλλέξουμε τις σχέσεις αντικειμένων-δράσεων μέσω κειμενικών δεδομένων που περιέχονται στις επισημειώσεις αυτού. Τέλος, τροποποιώντας τις επισημειώσεις ενός συγγενικού συνόλου δεδομένων \cite{amin_2013} για αναγνώριση πόζας, εξαγάγαμε επισημειώσεις για τις θέσεις των χεριών, οπότε μπορέσαμε να χτίσουμε μια βάση εκπαίδευσης ανιχνευτή χεριών.

\par Συνεχίσαμε με την υλοποίηση του προτεινόμενου συστήματος παρουσιάζοντας την κατασκευή κάθε υπομονάδας ξεχωριστά, αφού όπως τονίζουμε, ο σχεδιασμός επιτρέπει την αυτόνομη ενημέρωση ενός υποσυστήματος. Χρησιμοποιήσαμε τη μέθοδο των Πυκνών Τροχιών, συνδυαζόμενη με χαρακτηριστικά στάσης σώματος, για εξαγωγή χαρακτηριστικών Όρασης χαμηλού επιπέδου (low-level features). Ταυτόχρονα, συνδυάσαμε πληροφορία από την οπτική ανίχνευση αντικειμένων με αντίστοιχη από τη μελέτη υποτίτλων για να εξάγουμε τα αντικείμενα που εμφανίζονται και χρησιμοποιούνται στη διάρκεια του βίντεο και χρησιμοποιήσαμε την εμφάνιση των αντικειμένων ως χαρακτηριστικό αναπαράστασης των δράσεων. Κατά την ανίχνευση των αντικειμένων εισαγάγαμε την περιοχή ενδιαφέροντος, η οποία εκτείνεται γύρω από τον άνθρωπο και τις περιοχές κίνησης προσκηνίου. Θεωρήσαμε έγκυρες τις εμφανίσεις των αντικειμένων μόνο μέσα σε αυτή την περιοχή, ελαχιστοποιώντας έτσι την συμπερίληψη εμφάνισης αντικειμένων άσχετων με τη δράση στα χαρακτηριστικά. Δείξαμε ότι ο συνδυασμός υποτίτλων και οπτικής αντίληψης μπορεί να επιτύχει υψηλές επιδόσεις ανίχνευσης.

\par Στη συνέχεια εξαγάγαμε πληροφορία από τον τύπο λαβής (grasping type) και την χρησιμοποιήσαμε στην αναπαράσταση των δράσεων. Για το σκοπό αυτό υλοποιήσαμε έναν ανιχνευτή χεριών και τροφοδοτήσαμε την έξοδό του σε ένα συνελικτικό νευρωνικό δίκτυο (ResNet) στο οποίο είχε αφαιρεθεί το τελικό στάδιο. Έτσι, είδαμε το δίκτυο σαν μια μονάδα εξαγωγής συνελικτικών χαρακτηριστικών, τα οποία χρησιμοποιήσαμε αφενός για να ομαδοποιήσουμε τους τύπους λαβής σε 10 κατηγορίες και αφετέρου για να ταξινομήσουμε τις προκύπτουσες εικόνες χεριών σε κλάσεις. Δείξαμε ότι η εισαγωγή των ταξινομημένων τύπων λαβής ως χαρακτηριστικά στην περιγραφή των δράσεων βοηθάει ελαφρώς στην ταξινόμηση και δικαιολογήσαμε τη μικρή συνεισφορά με την απουσία ετικετών και επισημειώσεων για τύπους λαβής.

\par Για την ταξινόμηση, πειραματιστήκαμε με ποικίλες μεθόδους συνδυασμού χαρακτηριστικών και ταξινομητές. Μετά από σύγκριση διαφόρων αλγορίθμων, στο κομμάτι του ταξινομητή καταλήξαμε στις γραμμικές Μηχανές Διανυσμάτων Υποστήριξης (SVM) λόγω της ταχύτητας και της απόδοσης που πετύχαιναν. Όσον αφορά τον συνδυασμό, δοκιμάσαμε πρώτα τη δύναμη των χαμηλού επιπέδου χαρακτηριστικών, στη συνέχεια τα συνδυάσαμε με χαρακτηριστικά εμφάνισης αντικειμένων και τελικά με χαρακτηριστικά τύπων λαβής. Ταυτόχρονα, πειραματιστήκαμε με τους τρόπους συνδυασμού των διαφορετικών χαρακτηριστικών. Δείξαμε ότι όταν χρησιμοποιηθούν μόνα τους τα χαρακτηριστικά χαμηλού επιπέδου Όρασης, είναι ανάγκη να υποστούν ένα μη γραμμικό μετασχηματισμό, όπως ο $\chi^2$ πυρήνας, ενώ η συνένωση αποτυγχάνει. Με την εισαγωγή των χαρακτηριστικών εμφανίσεων αντικειμένων αιρείται αυτή η ανάγκη και το σχήμα Tf-Idf μπορεί να χρησιμοποιηθεί αντί όποιου άλλου μετασχηματισμού, οδηγώντας ταυτόχρονα σε καλύτερη απόδοση και επίδοση. Παρόμοια αποτελέσματα δίνει και η εισαγωγή των χαρακτηριστικών τύπων λαβής. Τέλος, εκμεταλλευτήκαμε τα δεδομένα κειμένου για να εξάγουμε σχέσεις αντικειμένων-δράσεων και να προσαρμόσουμε τις πιθανοτήτες εξόδου. Η πράξη αυτή αφενός ωθεί σε πολύ βελτιωμένες επιδόσεις και αφετέρου αίρει και την ανάγκη για κωδικοποίηση των διαφορετικών χαρακτηριστικών, αφού η απλή συνένωσή τους δίνει το καλύτερο αποτέλεσμα (73.1 mAP). Πειράματα θεωρώντας γνωστές τις εμφανίσεις των αντικειμένων έδειξαν ότι η μέθοδος αυτή μπορεί να συγκριθεί με τα state-of-the-art αποτελέσματα.

\par Εφαρμόσαμε τον ταξινόμητη που έδωσε το καλύτερο αποτέλεσμα αναγνώρισης που επιτύχαμε για να προβούμε σε κατάτμηση δράσεων. Προτείναμε έναν νέο αλγόριθμο κατάτμησης, ο οποίος εκκινεί από την αρχική τμήση με βάση τις μεταβολές των εμφανιζόμενων αντικειμένων και κάνει χρήση δυναμικού προγραμματισμού για τον περαιτέρω βέλτιστο διαχωρισμό των επιμέρους τμημάτων με βάση την ελαχιστοποίηση της τροποποιημένης SVM loss $1-(P_{max1}-P_{max2})$, με $P_{max1}$, $P_{max2}$ τις πιθανότητες της πιο πιθανής και της αμέσως πιο πιθανής κλάσης αντίστοιχα. Τα διαδοχικά τμήματα που προκύπτουν με ίδια κλάση συνενώνονται. Αξιολογούμε την επιτυχία του αλγορίθμου μας χρησιμοποιώντας ως μετρική την ανά frame mAP. Συγκεκριμένα, πετυχαίνουμε επίδοση 46.4 \% mAP ανά frame. Καθώς δεν υπάρχουν αποτελέσματα στη βιβλιογραφία για αυτό το πρόβλημα, δεν προβαίνουμε σε κάποια σύγκριση.

\par Πέραν των παραπάνω, στη διάρκεια της εργασίας συχνά σταθήκαμε σε ιστορικά στοιχεία σχετικά με την αναγνώριση και την κατάτμηση δράσεων με χρήση διαφορετικών καναλιών πληροφορίας, την ανίχνευση αντικειμένων σε εικόνες και την εξαγωγή προσκηνίου. Ταυτόχρονα, είδαμε θεωρητικά πολλές μεθόδους και αλγορίθμους, όπως η μέθοδος των Πυκνών Τροχιών και τα χαρακτηριστικά πόζας, η ανίχνευση αντικειμένων με χρήση Μοντέλων Παραμορφώσιμων Τμημάτων, η εξαγωγή προσκηνίου με χρήση Γκαουσιανών Μοντέλων Μίξης, τα Συνελικτικά Νευρωνικά Δίκτυα και πιο αναλυτικά το δίκτυο ResNet, τα χαρακτηριστικά BING και ο αλγόριθμος κατάτμησης των \cite{hoai_2011}. Μάλιστα στον τελευταίο, αναλύσαμε τα βήματά του και τη λειτουργία του και αναδείξαμε τις αδυναμίες του, οπότε στηριχθήκαμε σε αυτόν για να προτείνουμε έναν νέο αλγόριθμο κατάτμησης βίντεο. Τέλος, μελετήσαμε και παραθέσαμε μεθόδους και αποτελέσματα σχετικών εργασιών με τη δική μας, είτε για την επιρροή τους στη δική μας είτε για σύγκριση.

\par Συνοψίζοντας, οι κυριότερες συνεισφορές της εργασίας αυτής είναι οι εξής:
\begin{itemize}
	\item Η πρόταση ενός γενικευμένου ενιαίου συστήματος αναγνώρισης και κατάτμησης δράσεων σε αφαιρετική μορφή, σχεδιασμένο έτσι ώστε να αξιοποιεί πληροφορία πολλαπλών καναλιών και κάθε υπομονάδα του να είναι ανεξάρτητη από τις υπόλοιπες.
	\item Η υλοποίηση ενός τέτοιου συστήματος, η οποία καταλήγει σε απόδοση ταξινόμησης δράσεων που συγκρίνεται με τις κορυφαίες επιδόσεις της βιβλιογραφίας.
	\item Ο πειραματισμός με χαρακτηριστικά διαφορετικών καναλιών και η πειραματική απόδειξη της δύναμης του συνδυασμού πληροφορίας χαμηλού επιπέδου με σημασιολογική πληροφορία υψηλού επιπέδου.
	\item Ο πειραματισμός με διαφορετικές μεθόδους συνδυασμού χαρακτηριστικών και η πειραματική απόδειξη ότι με ισχυρά χαρακτηριστικά, όπως συνδυασμένα οπτικά και σημασιολογικά χαρακτηριστικά, αιρείται η ανάγκη για μη γραμμικό μετασχηματισμό και ότι αρκεί η συνένωση των χαρακτηριστικών σε ένα διάνυσμα που ταξινομείται με γραμμική SVM.
	\item Η ενίσχυση των πιθανοτήτων εξόδου του αλγορίθμου αναγνώρισης, η οποία φαίνεται να προσφέρει επιπλέον κέρδος στην απόδοση του ταξινομητή δράσεων.
	\item Η πρόταση και υλοποίηση ενός νέου αλγορίθμου κατάτμησης βίντεο δράσεων, ο οποίος κάνει χρήση πιθανοτήτων και δυναμικού προγραμματισμού.
	\item Η πρόταση και υλοποίηση ενός αλγορίθμου ανίχνευσης χεριών σε εικόνες, ο οποίος συνδυάζει χαρακτηριστικά BING, HOG και ιστογράμματα χρώματος και είναι αρκετά γρήγορος και εύρωστος.
	\item Η πρόταση και παροχή νέων επισημειώσεων στο σύνολο δεδομένων \cite{rohrbach_2012}. Ακόμα και αν οι επισημειώσεις αυτές δεν είναι άμεσα χρήσιμες, παραμένουν ως ένδειξη πιθανής μελλοντικής εργασίας πάνω σε ένα σύνολο δεδομένων που θα παρέχει, πέραν των επισημειωμένων βίντεο δράσεων, πληροφορία και για ανίχνευση αντικειμένων, χεριών και υποτίτλους ή ήχο.
\end{itemize}


\section{Προτάσεις Για Μελλοντική Έρευνα}
Καθώς η εργασία αυτή, όπως και κάθε εργασία, είναι πεπερασμένη, υπάρχουν ζητήματα τα οποία θα επιθυμήσουμε να εξετάσουμε σε μελλοντικές ερευνές μας. Πιστεύουμε ότι ορισμένες πρακτικές από αυτές είναι πολλά υποσχόμενες και μπορούν να αποδόσουν εξαιρετικά. Όπως εξάλλου τονίσαμε στη διάρκεια αυτής της εργασίας, η ανεξάρτητη σχεδίαση των υποσυστημάτων του προτεινόμενου συστήματος διευκολύνει την αυτόνομη ενημέρωση κάθε υπομονάδας ξεχωριστά, οπότε τα πειράματα που προτείνουμε μπορούν να γίνουν επηρεάζοντας μόνο το αντίστοιχο κομμάτι του συνολικού συστήματος.

\par Σε πρώτη φάση, εξετάζουμε πιθανές βελτιώσεις του υποσυστήματος οπτικής πληροφορίας χαμηλού επιπέδου. Η μέθοδος που χρησιμοποιήθηκε σε αυτή την εργασία είναι οι Πυκνές Τροχιές \cite{wang_2011}, συνδυαζόμενες με χαρακτηριστικά πόζας \cite{rohrbach_2012}. Ως φυσική συνέχεια βλέπουμε τη χρήση των Βελτιωμένων Πυκνών Τροχιών \cite{wang_2013}. Εξάλλου, δύο από τις εργασίες \cite{cherian_2017}, \cite{cheron_2015} που συγκρίνουμε με τη δική μας στο κεφάλαιο 6, πάνω στο ίδιο σύνολο δεδομένων, χρησιμοποιούν αυτή τη μέθοδο και κωδικοποιούν τα χαρακτηριστικά με Fisher Vectors, με την πρώτη εργασία μάλιστα να πετυχαίνει το τρέχον state-of-the-art αποτέλεσμα. Επιπλέον, οι δύο εργασίες αυτές χρησιμοποιούν Συνελικτικά Νευρωνικά Δίκτυα, για διαφορετικούς σκοπούς η κάθεμία. Μια ιδέα είναι να συνδυάσουμε συνελικτικά χαρακτηριστικά με χαρακτηριστικά Πυκνών Τροχιών, όπως στο \cite{cherian_2017}. Τέλος, ως βελτίωση πάνω στα χαρακτηριστικά πόζας, οι \cite{cheron_2015} χρησιμοποιούν νευρωνικά δίκτυα, κατεύθυνση που επίσης είναι ενδιαφέρουσα. Οι διαφορετικές μορφές χαρακτηριστικών μπορούν να συνδυαστούν με διαφορετικές μεθόδους σύμμειξης καναλιών πριν τον ταξινομητή.

\par Στην ανίχνευση αντικειμένων, θα ήταν χρήσιμο να δοκιμάσουμε διαφορετικούς αλγορίθμους ανίχνευσης, πιο εύρωστους και γενικούς από το Ταίριασμα Προτύπων. Επιπλέον, πιο σύνθετη λογική μπορεί να χρησιμοποιηθεί και στην ανάλυση των υποτίτλων, αφού κι εκεί εφαρμόζουμε γλωσσικό Ταίριασμα Προτύπων. Για παράδειγμα, μια ιδέα όσον αφορά την οπτική ανίχνευση είναι η εφαρμογή των Μοντέλων Παραμορφώσιμων Τμημάτων \cite{felzenszwalb_2008}. Μια άλλη ιδέα είναι ο συνδυασμός χαρακτηριστικών, όπως στην περίπτωση του ανιχνευτή χεριών. Το \cite{zhou_2015} εξάγει προτάσεις αντικειμένων με χαρακτηριστικά BING και παρακολουθεί τις περιοχές αυτές πυκνά στη διάρκεια του βίντεο. Στο \cite{malmaud_2015} εκπαιδεύεται ένας ανιχνευτής αντικειμένων (τροφών) με Συνελικτικά Νευρωνικά Δίκτυα. Στην ίδια εργασία, η ανάλυση υποτίτλων γίνεται αφού προηγηθεί συντακτική ανάλυση η οποία εξάγει τα μέρη του λόγου. Στη συνέχεια, οι συγγραφείς κρατάνε τα ουσιαστικά που εμφανίζονται γύρω από ένα ρήμα που επιλέγεται από μια γνωστή λίστα. Τέλος, το \cite{alayrac_2016}, χρησιμοποιεί συντακτικές σχέσεις ρήματος-αντικειμένου για να εξάγει κατηγορίες δράσεων με μη επιβλεπόμενο τρόπο, οι οποίες σχετίζονται άμεσα με μια κλάση αντικειμένων (π.χ. αλλαγή λάστιχου).

\par Σίγουρα μια σημαντική εργασία θα ήταν η εφαρμογή του συστήματός μας πάνω σε ένα άλλο σύνολο δεδομένων με διαφορετικές επισημειώσεις. Ιδανικά, θα θέλαμε επιπλέον, επισημειώσεις για ανίχνευση αντικειμένων και χεριών/τύπων λαβής, καθώς και υπότιτλους ή έστω ηχητικό κανάλι για να τους εξάγουμε. Στο κομμάτι της εξαγωγής τύπων λαβής ειδικά, θα μπορούσαμε σε αυτή την περίπτωση να διαπιστώσουμε πειραματικά τη συνεισφορά της πληροφορίας αυτής στην αναγνώριση των δράσεων, αφού όπως είδαμε, στην περίπτωσή μας είναι αρκετά μικρή, κάτι που δικαιολογήσαμε μέσω της εκπαίδευσης με τα αποτελέσματα ομαδοποίησης, χωρίς δηλαδή ετικέτες. Στο κομμάτι των υποτίτλων, θα είχε ενδιαφέρον η χρήση πραγματικών υποτίτλων, χωρίς τους περιορισμούς που εισαγάγαμε στη δημιουργία τους. Τέλος, η εφαρμογή του αλγορίθμου κατάτμησης που προτείνουμε θα είχε ενδιαφέρον σε ένα σύνολο δεδομένων όπου υπάρχουν και άλλα αποτελέσματα από σχετικές εργασίες στην παγκόσμια βιβλιογραφία, έτσι ώστε να μπορούμε να συγκρίνουμε τα αποτελέσματά μας.

\par Τέλος, στην περίπτωση της κατάτμησης, θεωρήσαμε σιωπηρά ότι τα χαρακτηριστικά εμφανίσεων αντικειμένων μένουν πρακτικά σταθερά σε μεγάλα τμήματα βίντεο. Η υπόθεση αυτή δε μας ενόχλησε, αφού χειριστήκαμε την κατάτμηση με τις ground truth επισημειώσεις εμφάνισης αντικειμένων. Εν τούτοις, αν επιχειρούσμε να κάνουμε το ίδιο με τα υπολογισμένα ιστογράμματα εμφανίσεων των αντικειμένων μέσω ανίχνευσης, τα αποτελέσματα θα ήταν διαφορετικά, καθώς θα υπάρχει αρκετός θόρυβος. Αν για παράδειγμα σε ένα frame εντοπισθεί λανθασμένα ένα αντικείμενο, τότε θα έχουμε σπάσιμο του βίντεο εκεί. Το πρόβλημα αυτο λύνεται μερικώς με την τελική συνένωση των διαδοχικών περιοχών που ταξινομήθηκαν στην ίδια κλάση. Από την άλλη, είναι ανάγκη να μελετηθούν μέθοδοι αποθορυβοποίησης των ιστογραμμάτων αντικειμένων. Η εισαγωγή Κρυφών Μαρκοβιανών Μοντέλων (ΗΜΜ) θα μπορούσε να συμβάλλει προς την κατεύθυνση αυτή.

\par Κλείνοντας, θα θέλαμε να επισημάνουμε ότι, καθώς η τεχνολογία εξελίσσεται ραγδαία, η παρούσα εργασία αναπόφευκτα θα ξεπεραστεί. Ευχόμαστε καλοπροαίρετα να γίνει αυτό, όμως ταυτόχρονα ελπίζουμε ότι η εργασία μας θα συνεισφέρει στην εξέλιξη κι ότι πάνω της μπορούν να βασιστούν νέες ερευνητικές πορείες που θα αποδειχθούν ισχυρότερες.

\end{document}
