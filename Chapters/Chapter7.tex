% !TEX root = ../Thesis.tex
% !TEX output_directory
\documentclass[11pt,a4paper,english,greek,twoside]{../Thesis}

\begin{document}
\chapter{Σύνοψη - Μελλοντική Εργασία} \label{chap:last}

\section{Σύνοψη}
εδω τι βάζω ακριβώς ? ενα πιο διανθισμένο abstract ? 

\section{Μελλοντική Εργασία}

\subsection{Εγκεφαλογράφος}
Παρότι ο EPOC δίνει ικανοποιητικά αποτελέσματα, οι διάφορες ασυνέπειες που παρουσιάζει τόσο όσον αφορά την ασταθή διάταξη των ηλεκτροδίων του, όσον και την ποιότητα των σημάτων του, θέτουν ένα άνω όριο όσον αφορά τις επιδόσεις του. Τα προβλήματα που μας παρουσιάστηκαν κατά την καταγραφή των σημάτων ήταν πάρα πολλά. Δεν ήταν λίγες οι φορές όπου προκειμένου να πετύχουμε καλή ποιότητα σήματος, περνούσε πολύ ώρα, καθιστώντας την καταγραφή των σημάτων μια επίπονη διαδικασία. Επιπλέον, σε δύο δοκιμές που κάναμε σε γυναίκες δεν καταφέραμε να λάβουμε εγκεφαλικά σήματα λόγω των πυκνών μαλλιών, που εμπόδιζαν την επίτευξη καλής ποιότητας επαφής. Ακόμα όμως και όταν επιτυγχάναμε καλή ποιότητα επαφής (σύμφωνα με το quality που επιστρέφει ο epoc), αυτό δεν αντικατοπτριζόταν στα σήματα. Ενδεικτικά, στην εικόνα \ref{fig:epoc_gui}, στο a) σκέλος, ενώ τα Άλφα κύματα που καταγράφει το O1, είναι πολύ πιο ισχυρά και με μεγαλύτερο SNR, παρατηρούμε πως αυτό δεν αντικατοπτρίζεται στους δείκτες ποιότητας, καθώς το κανάλι O2 έχει πολύ καλύτερο δείκτη π από το O1. Επιπλέον, δεν είναι καθόλου σαφές από την εταιρεία, τι ακριβώς δείχνει αυτός ο δείκτης. Σύμφωνα με το τεχνικό επιτελείο της εταιρείας, το πράσινο χρώμα ενός ηλεκτροδίου, δηλώνει αντίσταση επαφής μικρότερη των 200kΩ, πράγμα το οποίο δεν είναι καθόλου ενθαρρυντικό αν σκεφτεί κάποιος πως σε άλλα συστήματα, αντίσταση επαφής άνω των 20kΩ, άνω του επιτρεπτού ορίου για την καταγραφή EEG.
\par Ίσως να αξίζει να δοκιμαστούν και άλλοι low badget εγκεφαλογράφοι, όπως το EEG σύστημα που αναπτύχθηκε από την OpenBCI, που είναι μια εταιρεία ανοιχτού λογισμικού, η οποία κατασκευάζει ολοκληρωμένα συστήματα EEG (ενισχυτής, headset, ηλεκτρόδια). Το headset είναι εξολοκλήρου τυπωμένο σε 3D εκτυπωτή, και ο τρόπος με τον οποίο εφαρμόζει στο κρανίο, δεν αφήνει περιθώρια λάθους τοποθέτησης των ηλεκτροδίων. Όσον αφορά το κόστος πρέπει να ληφθεί υπόψιν η εξής παράμετρος. Στην οικονομική έκδοση του, το EPOC δεν παρέχει δυνατότητα λήψης των αρχικών εγκεφαλικών σημάτων (raw EEG), ενώ στην πλήρη έκδοσή του αυτή η δυνατότητα είναι περιορισμένη, καθώς απαιτείται μηνιαία συνδρομή. Συνεπώς, το σύστημα της OpenBCI παρότι κοστίζει λίγο παραπάνω, είναι οικονομικότερο σε βάθος χρόνου, ειδικά αν κάποιος συμπεριλάβει και την διαφορά ποιότητας μεταξύ τους, καθώς το σύστημα της OpenBCI μπορεί να συναγωνιστεί σε επιδόσεις διάσημους εγκεφαλογράφους ιατρικών προδιαγραφών, όπως ο g.USBamp \cite{Frey2016-fx}.

\subsection{Υλοποίηση Hardware}
Όσον αφορά το κατασκευαστικό μέρος (LEDs, Driver Circuit, Βάση), μείναμε απόλυτα ευχαριστημένοι από την λειτουργία του και το μόνο που θα πρέπει να ελέγξουμε μελλοντικά, είναι το κατά πόσο χρειάζονται και τα 20 led που χρησιμοποιήσαμε για κάθε οπτική διέγερση και αν μπορούν να ελαττωθούν ακόμα και κατά 1/4. Ένας τρόπος είναι να αρχίσουμε να τα ελαττώνουμε δοκιμάζοντας παράλληλα την επίδοση του συστήματος, έχοντας ως αναφορά τα αποτελέσματα που παρουσιάσαμε. Επιπλέον, μια κατασκευή συστοιχιών με RGB LEDs, παρότι θα αύξανε σημαντικά την πολυπλοκότητα της κατασκευής, θα επέτρεπε την γρήγορη και εύκολη μελέτη της επίδρασης του χρώματος της ΕΟΔ στα SSVEP σήματα. 

\subsection{Μεθοδολογικά - Διαδικασία καταγραφής}
Πιστεύουμε πως το πρωτόκολλο καταγραφής που εφαρμόσαμε ήταν πολύ καλά σχεδιασμένο, και βοήθησε στην λήψη σωστών δεδομένων εύκολα και γρήγορα, είναι γεγονός όμως πως στο μέλλον, προκειμένου να αποκτήσουν τα αποτελέσματα μας μεγαλύτερη βαρύτητα, θα πρέπει να γίνει καταγραφή σε περισσότερους από δύο χρήστες. Παρότι υπάρχουν δημοσιευμένες εργασίες οι οποίες κάνουν εξαγωγή συμπερασμάτων βασιζόμενες μόνο σε έναν χρήστη, τις περισσότερες φορές χρησιμοποιούνται 5-10 χρήστες.

\subsection{Υπολογιστικές μέθοδοι}

%\par Ένας άλλος τρόπος να αντιμετωπιστεί το πρόβλημα των θέσεων των ηλεκτροδίων, είναι να πραγματοποιηθεί το στάδιο της online δοκιμής, αμέσως μετά από αυτό της offline καταγραφής δεδομένων και εκπαίδευσης του συστήματος. Με αυτό τον τρόπο δεν θα έχουμε μετατόπιση των αισθητήρων. Προκύπτουν όμως δύο νέα προβλήματα. Πρώτον, η διαδικασία της offline καταγραφής μπορεί να φτάσει μέχρι και τα 40 λεπτά, που σημαίνει πως ο χρήστης είναι πιθανό να νιώσει κούραση και να μην είναι ικανός να συνεχίσει στο online σκέλος, και δεύτερον, δεν πρέπει να ξεχνάμε πως τα ηλεκτρόδια είναι υγρού τύπου, συνεπώς υπάρχει φόβος να στεγνώσει το υγρό επαφής, και να απαιτείται επανεφαρμογή του, η οποία πιθανόν να επηρεάσει την θέση των αισθητήρων.\cite{noauthor_undated-vj}

\par Η μέθοδος CCA έδωσε πολύ ικανοποιητικά και αξιόπιστα αποτελέσματα σχεδόν σε όλες τις περιπτώσεις. Μια άλλη κατεύθυνση σίγουρα αξίζει να κινηθεί μελλοντικά αυτή η εργασία, είναι η δοκιμή των διαφόρων παραλλαγών της CCA μεθόδου (CCA-Variants), που αναφέραμε και στο κεφάλαιο \ref{chap:survey}. Η ανάγκη για παραλλαγές της CCA προέκυψε καθώς ένα μειονέκτημα αυτής, είναι πως ο πίνακας προτύπων $Y$ (εξίσωση \eqref{eq:cca_templates_Y}) περιέχει ημιτονοειδή σήματα, τα οποία προσεγγίζουν, αλλά δεν ανταποκρίνονται πλήρως στην πραγματική μορφή των SSVEP σημάτων. Ένα παράδειγμα είναι ο IT-CCA (Individual Template CCA) \cite{Bin2011-eh}, οπού ο πίνακας προτύπων $Y$ είναι ξεχωριστός για κάθε χρήστη, και προκύπτει από δικά του SSVEP σήματα τα οποία λήφθηκαν κατά την περίοδο της εκπαίδευσης. Συγκεκριμένα στην δημοσίευση \cite{Nakanishi2015-md}, παρουσιάζονται αναλυτικά και συγκρίνονται έξι διαφορετικές μέθοδοι βασισμένες στην CCA, και καταλήγουν πως μια συνδυαστική μέθοδος CCA και IT-CCA βελτιώνει τα ποσοστά κατά $40-50\%$ για χρονικά παράθυρα μικρότερα του 1s, και περίπου $10\%$ για μεγαλύτερα. 

\par Ακόμα, θα μπορούσαμε να υλοποιήσουμε έναν πιο έξυπνο τρόπο για την αυτόματη απόρριψη των epoch (offline ανάλυση) ή αλλιώς χρονικών παραθύρων (online διεπαφή), κάνοντας χρήση των δεδομένων ποιότητας επαφής που παρέχει το EPOC για κάθε αισθητήρα κάθε στιγμή. Στον αλγόριθμό μας συλλέγουμε κανονικά αυτά τα δεδομένα, και τα αποθηκεύουμε σε δομές παρόμοιες με αυτές των τιμών του κάθε αισθητήρα (\ref{fig:epoch_matrix}), χωρίς όμως να τα χρησιμοποιούμε. Επιπλέον, θα μπορούσαμε ακόμα να λαμβάνουμε τις μετρήσεις από το επιταχυνσιόμετρο του EPOC, έτσι ώστε να εντοπίζουμε αμέσως τις χρονικές στιγμές στις οποίες ο χρήστης, για παράδειγμα, κούνησε απότομα το κεφάλι του, προκαλώντας παράσιτα στο εγκεφαλογράφημα.

\subsection{Online Διεπαφή}
\par Σίγουρα το κομμάτι του online συστήματος επιδέχεται πολλές βελτιώσεις οι οποίες έχουν να κάνουν κυρίως με την βελτίωση της αναγνώρισης της NC κατάστασης. Όσον αφορά την CCA, μια προσέγγιση, θα ήταν η καταγραφή ενός μεγάλου αριθμού από NC epochs, των οποίων ο μέσος όρος να προστεθεί στον template πίνακα Y (που περιέχει ήδη τα reference σήματα για κάθε ΕΟΔ). Με αυτό τον τρόπο η CCA θα μπορεί να συσχετίσει ένα νέο test σήμα, με την κατάσταση NC, με τον ίδιο ακριβώς τρόπο που κάνει για τα SSVEP σήματα. 

\par Επιπλέον, προκειμένου να γίνει μια σύγκριση της διεπαφής σε σχέση και με άλλες αντίστοιχες διεπαφές, θα πρέπει να δημιουργήσουμε μια δοκιμασία (όπως για παράδειγμα κάναμε με τον λαβύρινθο) η οποία να έχει χρησιμοποιηθεί και σε άλλες εργασίες (για παράδειγμα έναν speller) έτσι ώστε να μπορέσουμε να αποκτήσουμε μια καλύτερη εικόνα της επιδόσεις του συστήματος μας, σε σχέση με άλλα. 

\par Τέλος, ενδιαφέρον θα είχε η χρήση και άλλων modalities σε συνδυασμό με τα EEG σήματα, για την δημιουργία υβριδικών διεπαφών. Από τα σήματα του επιταχυνσιόμετρου του Epoc μπορεί να εξαχθούν πληροφορίες για την κατεύθυνση του κεφαλιού του χρήστη. Επιπλέον, θα μπορούσε να χρησιμοποιηθεί η κάμερα του υπολογιστή έτσι ώστε να εξάγουμε προσεγγιστικά την κατεύθυνση είτε του κεφαλιού είτε των ματιών. Μια τέτοια πολυτροπική (multimodal) προσέγγιση του προβλήματος θα μπορούσε να ανοίξει νέες κατευθύνσεις στις χρήσεις των BCI.


\end{document}