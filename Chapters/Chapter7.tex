% !TEX root = ../Thesis.tex
% !TEX output_directory
\documentclass[11pt,a4paper,english,greek,twoside]{../Thesis}

\begin{document}
\chapter{Μελλοντική Εργασία - Σύνοψη} \label{chap:last}

\section{Μελλοντική Εργασία}

\subsection{Εγκεφαλογράφος}
Παρότι ο EPOC δίνει ικανοποιητικά αποτελέσματα, οι διάφορες ασυνέπειες που παρουσιάζει τόσο όσον αφορά την ασταθή διάταξη των ηλεκτροδίων του, όσον και την ποιότητα των σημάτων του, θέτουν ένα άνω όριο όσον αφορά τις επιδόσεις του. Ίσως να αξίζει να δοκιμαστούν και άλλοι low badget εγκεφαλογράφοι, όπως το EEG σύστημα που αναπτύχθηκε από την OpenBCI, που είναι μια εταιρεία ανοιχτού λογισμικού, η οποία κατασκευάζει ολοκληρωμένα συστήματα EEG (ενισχυτής, headset, ηλεκτρόδια). Το headset είναι εξολοκλήρου τυπωμένο σε 3D εκτυπωτή, και ο τρόπος με τον οποίο εφαρμόζει στο κρανίο, δεν αφήνει περιθώρια λάθος τοποθέτησης των ηλεκτροδίων. Όσον αφορά το κόστος πρέπει να ληφθεί υπόψην η εξής παράμετρος. Στην οικονομική έκδοση του, το EPOC δεν παρέχει δυνατότητα λήψης των αρχικών εγκεφαλικών σημάτων (raw EEG), ενώ στην πλήρη έκδοσή του αυτή η δυνατότητα είναι περιορισμένη, καθώς απαιτείται μηνιαία συνδρομή. Συνεπώς, το σύστημα της OpenBCI παρότι κοστίζει λίγο παραπάνω, είναι οικονομικότερο σε βάθος χρόνου, ειδικά αν κάποιος συμπεριλάβει και την διαφορά ποιότητας μεταξύ τους, καθώς το σύστημα της OpenBCI μπορεί να συναγωνιστεί σε επιδόσεις διάσημους εγκεφαλογράφους ιατρικών προδιαγραφών, όπως ο g.USBamp \cite{Frey2016-fx}.

\subsection{Υλοποίηση Hardware}
Όσον αφορά το κατασκευαστικό μέρος (LEDs, Driver Circuit, Βάση), μείναμε απόλυτα ευχαριστημένοι από την λειτουργία του και το μόνο που θα πρέπει να ελέγξουμε μελλοντικά, είναι το κατά πόσο είναι περιττά τα 20 led που χρησιμοποιήσαμε για κάθε οπτική διέγερση και αν μπορούν να ελαττωθούν ακόμα και στο 1/4. Ένας τρόπος είναι να αρχίσουμε να τα ελαττώνουμε δοκιμάζοντας παράλληλα την επίδοση του συστήματος, έχοντας ως αναφορά τα αποτελέσματα που παρουσιάσαμε.

\subsection{Μεθοδολογικά - Διαδικασία καταγραφής}
Πιστεύουμε πως το πρωτόκολλο καταγραφής που εφαρμόσαμε ήταν πολύ καλά σχεδιασμένο, και βοήθησε στην λήψη σωστών δεδομένων εύκολα και γρήγορα, είναι γεγονός όμως πως στο μέλλον, προκειμένου να αποκτήσουν τα αποτελέσματα μας μεγαλύτερη βαρύτητα, θα πρέπει να γίνει καταγραφή σε περισσότερους από δύο χρήστες. Παρότι υπάρχουν δημοσιευμένες εργασίες οι οποίες κάνουν εξαγωγή συμπερασμάτων βασιζόμενες μόνο σε έναν χρήστη, τις περισσότερες φορές χρησιμοποιούνται 5-10 χρήστες.

\subsection{Υπολογιστικές μέθοδοι}
Όπως αναφέραμε στην παράγραφο ???? η μη σταθερότητα των θέσεων των ηλεκτροδίων προκαλεί προβλήματα, τα οποία θα μπορούσαν ίσως να αντιμετωπιστούν ενσωματώνοντάς διάφορες μεθόδους χωρικού φιλτραρίσματος εγκεφαλογραφήματος στο στάδιο της προ-επεξεργασίας του σήματος (CSP, MEC κ.α). Με αυτό τον τρόπο η μέθοδος PSD-PCA-MLR-kNN που έδωσε τα καλύτερα αποτελέσματα κατά την offline επεξεργασία, θα μπορούσε να δώσει καλύτερα αποτελέσματα στην online.

\par Ένας άλλος τρόπος να αντιμετωπιστεί το πρόβλημα των θέσεων των ηλεκτροδίων, είναι να πραγματοποιηθεί το στάδιο της online δοκιμής, αμέσως μετά από αυτό της offline καταγραφής δεδομένων και εκπαίδευσης του συστήματος. Με αυτό τον τρόπο δεν θα έχουμε μετατόπιση των αισθητήρων. Προκύπτουν όμως δύο νέα προβλήματα. Πρώτον, η διαδικασία της offline καταγραφής μπορεί να φτάσει μέχρι και τα 40 λεπτά, που σημαίνει πως ο χρήστης είναι πιθανό να νιώσει κούραση και να μην είναι ικανός να συνεχίσει στο online σκέλος, και δεύτερον, δεν πρέπει να ξεχνάμε πως τα ηλεκτρόδια είναι υγρού τύπου, συνεπώς υπάρχει φόβος να στεγνώσει το υγρό επαφής, και να απαιτείται επανεφαρμογή του, η οποία πιθανόν να επηρεάσει την θέση των αισθητήρων.\cite{noauthor_undated-vj}

\par Μια άλλη κατεύθυνση στην οποία θα μπορούσε να κινηθεί μελλοντικά αυτή η εργασία, είναι η δοκιμή των διαφόρων παραλλαγών της CCA μεθόδου (CCA-Variants). Η ανάγκη για παραλλαγές της CCA προέκυψε καθώς ένα μειονέκτημα αυτής, είναι πως ο πίνακας προτύπων $Y$ (εξίσωση ΤΑΔΕ) περιέχει ημιτονοειδή σήματα, τα οποία προσεγγίζουν, αλλά δεν ανταποκρίνονται πλήρως στην πραγματική μορφή των SSVEP σημάτων. Παραλαγές όπως ο Multi-way CCA \cite{Zhang2011-uf} ή ο IT-CCA (Individual Template CCA) \cite{Bin2011-eh}, οπού ο πίνακας προτύπων $Y$ είναι ξεχωριστός για κάθε χρήστη, και προκύπτει από δικά του SSVEP σήματα τα οποία λήφθηκαν κατά την περίοδο της εκπαίδευσης. Συγκεκριμένα στην δημοσίευση \cite{Nakanishi2015-md}, συγκρίνονται έξι διαφορετικές μέθοδοι βασισμένες στην CCA, και καταλήγει πως μια συνδυαστική μέθοδος CCA και IT-CCA βελτιώνει τα ποσοστά κατά $40-50\%$ για χρονικά παράθυρα μικρότερα του 1s, και περίπου $10\%$ για μεγαλύτερα. 

\par Τέλος, θα μπορούσαμε να υλοποιήσουμε έναν πιο έξυπνο τρόπο για την αυτόματη απόρριψη των epoch (offline ανάλυση) ή αλλιώς χρονικών παραθύρων (online διεπαφή), κάνοντας χρήση των δεδομένων ποιότητας επαφής που παρέχει το EPOC για κάθε αισθητήρα κάθε στιγμή. Στον αλγόριθμό μας συλλέγουμε κανονικά αυτά τα δεδομένα, και τα αποθηκεύουμε σε δομές παρόμοιες με αυτές των τιμών του κάθε αισθητήρα (ΠΑΡΑΓΡΑΦΟΣ EPOCHING), χωρίς όμως να τα χρησιμοποιούμε. Επιπλέον, θα μπορούσαμε ακόμα να λαμβάνουμε τις μετρήσεις από το επιταχυνσιόμετρο του EPOC, έτσι ώστε να εντοπίζουμε αμέσως τις χρονικές στιγμές στις οποίες ο χρήστης, για παράδειγμα, κούνησε απότομα το κεφάφλι του, προκαλώντας παράσιτα στο εγκεφαλογράφημα.

\subsection{Online Διεπαφή}
\par Σίγουρα το κομμάτι του online συστήματος επιδέχεται πολλές βελτιώσεις. πες για την NC οτι οκ ολα καλα, αλλα πρεπει να ψαχτουμε με πιο εξυπνα κριτηρια

\par Τέλος, προκειμένου να γίνει μια σύγκριση της διεπαφής σε σχέση και με άλλες αντίστοιχες διεπαφές, θα πρέπει να δημιουργήσουμε μια δοκιμασία (όπως για παράδειγμα κάναμε με τον λαβύρινθο) η οποία να έχει χρησιμοποιηθεί και σε άλλες εργασίες (για παράδειγμα έναν speller) έτσι ώστε να μπορέσουμε να εξάγουμε ασφαλή αποτελέσματα.

\section{Σύνοψη}

\end{document}