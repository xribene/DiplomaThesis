% !TEX root = ../Thesis.tex
% !TEX output_directory
\documentclass[11pt,a4paper,english,greek,twoside]{../Thesis}
\newcommand{\rpm}{\raisebox{.2ex}{$\scriptstyle\pm$}}
\begin{document}
\chapter{Σύνοψη - Μελλοντική Εργασία} \label{chap:last}

\section{Σύνοψη}

\par Σε αυτή την διπλωματική εργασία, ασχοληθήκαμε με την μελέτη των οπτικών προκλητών δυναμικών σταθερής κατάστασης (SSVEP), και κατά πόσο μπορούμε να υλοποιήσουμε μια διεπαφή εγκεφάλου-υπολογιστή (BCI) βασισμένη σε αυτά τα δυναμικά, κάνοντας χρήση του εγκεφαλογράφου EPOC, ενός low-badget εγκεφαλογράφου σχεδιασμένο από την εταιρεία Emotiv, για την χρήση του σε non-critical εφαρμογές. Πέρα από τον εγκεφαλογράφο, προσπαθήσαμε να ρυθμίσουμε βέλτιστα όλα τα άλλα υποσυστήματα της διεπαφής, προκειμένου να αντισταθμίσουμε τα μέτριας ποιότητας σήματα που αναμέναμε να λάβουμε από τον Epoc.

\par Αρχικά, έπρεπε να υλοποιήσουμε τον μηχανισμό παραγωγής των επαναλαμβανόμενων οπτικών διεγέρσεων (ΕΟΔ), με τέτοιο τρόπο ώστε να είναι αξιόπιστος, και ευέλικτος προκειμένου να αλλάζουμε ξεχωριστά και εύκολα τις παραμέτρους της κάθε ΕΟΔ (συχνότητα, duty cycle). Παρότι, οι ΕΟΔ συνήθως δημιουργούνται προγραμματιστικά στην οθόνη ενός υπολογιστή, εναλλακτικoί τρόποι, όπως η χρήση LED, φαίνεται να προκαλούν ισχυρότερα SSVEP σήματα. Στην υποενότητα \ref{subsec:leds}, περιγράψαμε αναλυτικά τον σχεδιασμό και την κατασκευή του μηχανισμού, ο οποίος περιλαμβάνει τέσσερις LED συστοιχίες, την βάση τους για την οθόνη του υπολογιστή, το κύκλωμα οδήγησής τους καθώς και το λογισμικό ελέγχου τους, μέσω του υπολογιστή. 

\par Στην συνέχεια, προκειμένου να γίνει η offline καταγραφή των σημάτων, υλοποιήθηκε ένα πρωτόκολλο καταγραφής \ref{subsec:protocol}, το οποίο ακολουθήθηκε πιστά σε κάθε καταγραφή για κάθε άτομο, έτσι ώστε να εξαλειφθεί η όποια ασυνέπεια μεταξύ των πειραματικών δοκιμών. Ο σχεδιασμός του, έγινε με τέτοιον τρόπο, έτσι ώστε να προσομοιώνεται όσο τον δυνατόν καλύτερα η λειτουργία πραγματικού χρόνου (online), και περιλάμβανε τόσο ηχητικές, όσο και οπτικές ενδείξεις (cues), για να καταφέρει ο χρήστης να ολοκληρώσει με επιτυχία την διαδικασία.

\par Σε επόμενο στάδιο, παρουσιάσαμε τα αποτελέσματα αυτής της εργασίας, για δύο χρήστες, όπου ο ένας μόνο είχε πρότερη εμπειρία χρήσης ενός παρόμοιου συστήματος. Η μέθοδος CCA έδωσε τα καλύτερα αποτελέσματα και για τους δύο χρήστες, με μέση πιστότητα (Accuracy) 89\% και ρυθμό ITR 19.05 bits/min. Επιπλέον, μελετήσαμε την δυνατότητα του συστήματος να ανιχνεύει την εντολή No Control (NC), δηλαδή την κατάσταση στην οποία ο χρήστης δεν παρατηρεί κάποια ΕΟΔ. Και πάλι η CCA πέτυχε την υψηλότερη επίδοση (75.5\% μέση πιστότητα με ITR 10.88 bits/sec) καθιστώντας την, την πιο κατάλληλη μέθοδο για την υλοποίηση του online σκέλους της διεπαφής. Τέλος, προκειμένου να αυξήσουμε τις εντολές που μπορεί να παράξει το online σύστημα, προσπαθήσαμε να διαχωρίσουμε τις καταστάσεις όπου ο χρήστης έχει ανοιχτά και κλειστά μάτια, πετυχαίνοντας μέσο ποσοστό πιστότητας 97.5\% και για τους δύο χρήστες.

\par Για το online σκέλος υλοποιήσαμε δύο διεπαφές, μια asynchronous (συμπεριλαμβάνεται η NC) και μία constant-enganged (χωρίς NC). Oι δύο χρήστες κλήθηκαν να ολοκληρώσουν μια δοκιμασία πλοήγησης ενός avatar μέσα σε έναν λαβύρινθο. Οι τέσσερις δυνατές κατευθύνσεις αντιστοιχήθηκαν στις τέσσερις ΕΟΔ, ενώ όταν ο χρήστης έκλεινε τα μάτια του, γινόταν αναίρεση της προηγούμενης εντολής. Ο μέσος χρόνος ολοκλήρωσης της δοκιμασίας ήταν 152 ± 9sec (asynchronous), και 89 ± 15sec (constantly engaged). Λόγω της έλλειψής NC κατάστασης, στην constantly engaged διεπαφή, οι χρήστες χρειάστηκαν λιγότερο χρόνο για την ολοκλήρωση της δοκιμασίας, ωστόσο την περιέγραψαν ως περισσότερο πιεστική, δηλώνοντας την προτίμηση τους προς την asynchronous διεπαφή.

\par Τα πρώτα προκαταρκτικά αποτελέσματα που παρουσιάσαμε για αυτούς τους δύο χρήστες, επιβεβαίωσαν πως τα SSVEP σήματα μπορούν να παραχθούν εύκολα στον εγκέφαλο ενός χρήστη, χωρίς καθόλου εκπαίδευση από μέρος του, και έδειξαν πως ο EPOC μπορεί να χρησιμοποιηθεί για την υλοποίηση BCI-SSVEP διεπαφών. Ωστόσο, η γενική ασυνέπεια που παρουσίαζε ο EPOC στην ποιότητα των σημάτων και της σύνδεσης, καθ' όλη την διάρκεια των offline και online καταγραφών, επηρέασε σημαντικά τα αποτελέσματα, και αποτέλεσε σημαντικό εμπόδιο στην λήψη περισσότερων δεδομένων, προκειμένου να πειραματιστούμε με περισσότερες παραμέτρους και να εξαχθούν πιο ασφαλή συμπεράσματα.

\section{Μελλοντική Εργασία}

\subsection{Εγκεφαλογράφος}

Παρότι ο EPOC δίνει ικανοποιητικά αποτελέσματα, οι διάφορες ασυνέπειες που παρουσιάζει τόσο όσον αφορά την ασταθή διάταξη των ηλεκτροδίων του, όσον και την ποιότητα των σημάτων του, θέτουν ένα άνω όριο όσον αφορά τις επιδόσεις του. Τα προβλήματα που μας παρουσιάστηκαν κατά την καταγραφή των σημάτων ήταν πάρα πολλά. Δεν ήταν λίγες οι φορές όπου προκειμένου να πετύχουμε καλή ποιότητα σήματος, περνούσε πολύ ώρα, καθιστώντας την καταγραφή των σημάτων μια επίπονη διαδικασία. Επιπλέον, σε δύο δοκιμές που κάναμε σε θηλυκούς χρήστες, δεν καταφέραμε να λάβουμε εγκεφαλικά σήματα λόγω των πυκνών μαλλιών, που εμπόδιζαν την επίτευξη καλής ποιότητας επαφής. Ακόμα όμως και όταν επιτυγχάναμε καλή ποιότητα επαφής (σύμφωνα με το quality που επιστρέφει ο epoc), αυτό δεν αντικατοπτριζόταν στα σήματα. Ενδεικτικά, στην εικόνα \ref{fig:epoc_gui}, στο a) σκέλος, ενώ τα Άλφα κύματα που καταγράφει το O1, είναι πολύ πιο ισχυρά και με μεγαλύτερο SNR, παρατηρούμε πως αυτό δεν αντικατοπτρίζεται στους δείκτες ποιότητας, καθώς το κανάλι O2 έχει πολύ καλύτερο δείκτη ποιότητας από το O1. Επιπλέον, δεν είναι καθόλου σαφές, τι ακριβώς αντικατοπτρίζει αυτός ο δείκτης. Σύμφωνα με το τεχνικό επιτελείο της εταιρείας, το πράσινο χρώμα ενός ηλεκτροδίου, δηλώνει αντίσταση επαφής μικρότερη των 200kΩ, πράγμα το οποίο δεν είναι καθόλου ενθαρρυντικό αν σκεφτεί κανείς πως σε άλλα συστήματα, αντίσταση επαφής άνω των 20kΩ, θεωρείται άνω του επιτρεπτού ορίου για την σωστή καταγραφή EEG.
\par Ίσως να αξίζει να δοκιμαστούν και άλλοι low badget εγκεφαλογράφοι, όπως το EEG σύστημα που αναπτύχθηκε από την OpenBCI, που είναι μια εταιρεία ανοιχτού λογισμικού, η οποία κατασκευάζει ολοκληρωμένα συστήματα EEG (ενισχυτής, headset, ηλεκτρόδια). Το headset είναι εξολοκλήρου τυπωμένο σε 3D εκτυπωτή, και ο τρόπος με τον οποίο εφαρμόζει στο κρανίο, δεν αφήνει περιθώρια λάθους τοποθέτησης των ηλεκτροδίων. Όσον αφορά το κόστος πρέπει να ληφθεί υπόψιν η εξής παράμετρος. Στην οικονομική έκδοση του, το EPOC δεν παρέχει δυνατότητα λήψης των αρχικών εγκεφαλικών σημάτων (raw EEG), ενώ στην πλήρη έκδοσή του αυτή η δυνατότητα είναι περιορισμένη, καθώς απαιτείται μηνιαία συνδρομή. Συνεπώς, το σύστημα της OpenBCI παρότι κοστίζει λίγο παραπάνω, είναι οικονομικότερο σε βάθος χρόνου, ειδικά αν κάποιος συμπεριλάβει και την διαφορά ποιότητας μεταξύ τους, καθώς το σύστημα της OpenBCI μπορεί να συναγωνιστεί σε επιδόσεις διάσημους εγκεφαλογράφους ιατρικών προδιαγραφών, όπως ο g.USBamp \cite{Frey2016-fx}.

\subsection{Υλοποίηση Hardware}
Όσον αφορά το κατασκευαστικό μέρος (LEDs, Driver Circuit, Βάση), μείναμε απόλυτα ευχαριστημένοι από την λειτουργία του και το μόνο βασικό που θα πρέπει να ελέγξουμε μελλοντικά, είναι το κατά πόσο χρειάζονται και τα 20 led που χρησιμοποιήσαμε για κάθε οπτική διέγερση και αν μπορούν να ελαττωθούν ακόμα και κατά 1/4. Επιπλέον, μια κατασκευή συστοιχιών με RGB LEDs, παρότι θα αύξανε σημαντικά την πολυπλοκότητα της κατασκευής, θα επέτρεπε την γρήγορη και εύκολη μελέτη της επίδρασης του χρώματος της ΕΟΔ στα SSVEP σήματα. 

%\par Τέλος, όπως είδαμε και σε δημοσιεύσεις που αναφέρονται στο κεφάλαιο \ref{chap:survey}, χρησιμοποιώντας ΕΟΔ που πέρα από την συχνότητα, διαφέρουν και στην φάση, είναι δυνατόν να επιτευχθεί σημαντική βελτίωση στις επιδόσεις του συστήματος. Συνεπώς, στο μέλλον θα πρέπει να τροποποιηθεί περεταίρω η βιβλιοθήκη Timer.h 

\subsection{Μεθοδολογικά - Διαδικασία καταγραφής}
Πιστεύουμε πως το πρωτόκολλο καταγραφής που εφαρμόσαμε ήταν πολύ καλά σχεδιασμένο, και βοήθησε στην λήψη σωστών δεδομένων εύκολα και γρήγορα, ωστόσο, είναι γεγονός όμως πως στο μέλλον, προκειμένου να εξάγουμε ασφαλέστερα συμπεράσματα, θα πρέπει να γίνει καταγραφή σε περισσότερους από δύο χρήστες. Παρότι υπάρχουν δημοσιευμένες εργασίες οι οποίες κάνουν εξαγωγή συμπερασμάτων βασιζόμενες μόνο σε έναν χρήστη, τις περισσότερες φορές χρησιμοποιούνται 5-10 χρήστες.

\subsection{Υπολογιστικές μέθοδοι}

%\par Ένας άλλος τρόπος να αντιμετωπιστεί το πρόβλημα των θέσεων των ηλεκτροδίων, είναι να πραγματοποιηθεί το στάδιο της online δοκιμής, αμέσως μετά από αυτό της offline καταγραφής δεδομένων και εκπαίδευσης του συστήματος. Με αυτό τον τρόπο δεν θα έχουμε μετατόπιση των αισθητήρων. Προκύπτουν όμως δύο νέα προβλήματα. Πρώτον, η διαδικασία της offline καταγραφής μπορεί να φτάσει μέχρι και τα 40 λεπτά, που σημαίνει πως ο χρήστης είναι πιθανό να νιώσει κούραση και να μην είναι ικανός να συνεχίσει στο online σκέλος, και δεύτερον, δεν πρέπει να ξεχνάμε πως τα ηλεκτρόδια είναι υγρού τύπου, συνεπώς υπάρχει φόβος να στεγνώσει το υγρό επαφής, και να απαιτείται επανεφαρμογή του, η οποία πιθανόν να επηρεάσει την θέση των αισθητήρων.\cite{noauthor_undated-vj}

\par Η μέθοδος CCA έδωσε πολύ ικανοποιητικά και αξιόπιστα αποτελέσματα σχεδόν σε όλες τις περιπτώσεις. Μια άλλη κατεύθυνση σίγουρα αξίζει να κινηθεί μελλοντικά αυτή η εργασία, είναι η δοκιμή των διαφόρων παραλλαγών της CCA μεθόδου (CCA-Variants), που αναφέραμε και στο κεφάλαιο \ref{chap:survey}. Η ανάγκη για παραλλαγές της CCA προέκυψε καθώς ένα μειονέκτημα αυτής, είναι πως ο πίνακας προτύπων $Y$ (εξίσωση \eqref{eq:cca_templates_Y}) περιέχει ημιτονοειδή σήματα, τα οποία προσεγγίζουν, αλλά δεν ανταποκρίνονται πλήρως στην πραγματική μορφή των SSVEP σημάτων. Ένα παράδειγμα είναι ο IT-CCA (Individual Template CCA) \cite{Bin2011-eh}, οπού ο πίνακας προτύπων $Y$ είναι ξεχωριστός για κάθε χρήστη, και προκύπτει από δικά του SSVEP σήματα τα οποία λήφθηκαν κατά την περίοδο της εκπαίδευσης. Συγκεκριμένα στην δημοσίευση \cite{Nakanishi2015-md}, παρουσιάζονται αναλυτικά και συγκρίνονται έξι διαφορετικές μέθοδοι βασισμένες στην CCA, και καταλήγουν πως μια συνδυαστική μέθοδος CCA και IT-CCA βελτιώνει τα ποσοστά κατά $40-50\%$ για χρονικά παράθυρα μικρότερα του 1s, και περίπου $10\%$ για μεγαλύτερα. 

\par Ακόμα, θα μπορούσαμε να υλοποιήσουμε έναν πιο έξυπνο τρόπο για την αυτόματη απόρριψη των epoch (offline ανάλυση) ή αλλιώς χρονικών παραθύρων (online διεπαφή), κάνοντας χρήση των δεδομένων ποιότητας επαφής που παρέχει το EPOC για κάθε αισθητήρα κάθε στιγμή. Στον αλγόριθμό μας συλλέγουμε κανονικά αυτά τα δεδομένα, και τα αποθηκεύουμε σε δομές παρόμοιες με αυτές των τιμών του κάθε αισθητήρα (\ref{fig:epoch_matrix}), χωρίς όμως να τα χρησιμοποιούμε. Επιπλέον, θα μπορούσαμε ακόμα να λαμβάνουμε τις μετρήσεις από το επιταχυνσιόμετρο του EPOC, έτσι ώστε να εντοπίζουμε αμέσως τις χρονικές στιγμές στις οποίες ο χρήστης, για παράδειγμα, κούνησε απότομα το κεφάλι του, προκαλώντας παράσιτα στο εγκεφαλογράφημα.

\subsection{Online Διεπαφή}
\par Σίγουρα το κομμάτι του online συστήματος επιδέχεται τροποποιήσεις, οι οποίες έχουν να κάνουν κυρίως με την βελτίωση της αναγνώρισης της NC κατάστασης. Όσον αφορά την CCA, μια προσέγγιση, θα ήταν η καταγραφή ενός μεγάλου αριθμού από NC epochs, των οποίων ο μέσος όρος να χρησιμοποιηθεί για την δημιουργία ενός template πίνακα Y (αντίστοιχα με τον ημιτονοειδή Y για κάθε ΕΟΔ συχνότητα). Με αυτό τον τρόπο η CCA θα μπορεί να συσχετίσει ένα νέο test σήμα, με την κατάσταση NC, με τον ίδιο ακριβώς τρόπο που κάνει για τα SSVEP σήματα. 

\par Επιπλέον, προκειμένου να γίνει μια σύγκριση της διεπαφής σε σχέση και με άλλες αντίστοιχες διεπαφές, θα πρέπει να δημιουργήσουμε μια δοκιμασία (όπως για παράδειγμα κάναμε με τον λαβύρινθο) η οποία να έχει χρησιμοποιηθεί και σε άλλες εργασίες (για παράδειγμα έναν speller) έτσι ώστε να μπορέσουμε να αποκτήσουμε μια καλύτερη εικόνα της επίδοσης του συστήματος μας, σε σχέση με άλλα. 

\par Τέλος, ενδιαφέρον θα είχε η χρήση και άλλων modalities σε συνδυασμό με τα EEG σήματα, για την δημιουργία υβριδικών διεπαφών. Από τα σήματα του επιταχυνσιόμετρου του Epoc μπορεί να εξαχθούν πληροφορίες για την κατεύθυνση του κεφαλιού του χρήστη. Επιπλέον, θα μπορούσε να χρησιμοποιηθεί η κάμερα του υπολογιστή έτσι ώστε να εξάγουμε προσεγγιστικά την κατεύθυνση είτε του κεφαλιού είτε των ματιών. Μια τέτοια πολυτροπική (multimodal) προσέγγιση του προβλήματος θα μπορούσε να ανοίξει νέες κατευθύνσεις στις χρήσεις των BCI.


\end{document}