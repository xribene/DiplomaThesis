% !TEX root = ../Thesis.tex
% !TEX output_directory
\documentclass[11pt,a4paper,english,greek,twoside]{../Thesis}
\begin{document}
\chapter{Θεωριτικό Υπόβαθρο Ηλεκτροεγκεφαλογραφήματος (ΗΕΓ) και Σημάτων}
\section{Βιοηλεκτρικά Σήματα και ΗΕΓ}
%Στο εισαγωγικό κεφάλαιο εστιάσαμε στην σημασία της μελέτης του εγκεφάλου και των σημάτων που μπορούμε να εξάγουμε από αυτόν.
Τα βιοηλεκτρικά σήματα είναι το αποτέλεσμα των ηλεκτροχημικών μεταβολών που λαμβάνουν χώρα εντός και μεταξύ των κυττάρων των νεύρων καθώς και των μυών. Πιο συγκεκριμένα, εάν ένα τέτοιο κύτταρο δεχθεί ερέθισμα ισχυρότερο από ένα κατώφλι συνήθως μεταξύ  -55mV και -50mV, τότε θα παράγει ένα δυναμικό δράσης το οποίο θα μεταδοθεί και θα διεγείρει γειτονικά κύτταρα. Αυτή η ομαδική δραστηριότητα των κυττάρων παράγει ηλεκτρικά πεδία ικανά να ανιχνευτούν με την βοήθεια ηλεκτροδίων τα οποία τοποθετούνται στην επιφάνεια του αντίστοιχου οργάνου, είτε στην δερματική επιφάνεια πάνω αυτό το όργανο. Όταν το ζωτικό αυτό όργανο είναι ο εγκέφαλος τότε το βιοηλεκτρικό σήμα ονομάζεται Hλεκτροεγκεφαλογράφημα (ΗΕΓ – EEG) και η πρώτη καταγραφή ΗΕΓ, έγινε το 1924 από τον Γερμανό ψυχίατρο Hans Berger. pros cons

\section{Εγκεφαλογράφος}
Ο εγκεφαλογράφος είναι μια περίπλοκη συσκευή με πολλά υποσυστήματα που επιτελούν διαφορετικές λειτουργίες, και τα οποία θα αναλύσουμε στην συνέχεια. \cite{Meng2011ABCI}

Το πρώτο από αυτά αναφέρθηκε και πριν, είναι τα ηλεκτρόδια τα οποία μετατρέπουν το ιοντικό ρεύμα που δημιουργείται στην επιφάνεια του ιστού σε ρεύμα ηλεκτρονίων
Επειδή τα ηλεκτρικά πεδία που παράγονται από τον εγκέφαλο και ανιχνεύονται από τα ηλεκτρόδια, έχουν πολύ χαμηλό πλάτος (0.5 - 200µV), απαιτείται η χρήση ενισχυτή με μεγάλος κέρδος
   

\end{document}
