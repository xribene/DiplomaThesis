%% !TEX root = ../Thesis.tex
%% !TEX output_directory
\documentclass[11pt,a4paper,english,greek,twoside]{../Thesis}
\begin{document}
\chapter{Εισαγωγή}
\section{Σημασία Μελέτης Εγκεφάλου και Βιοηλεκτρικών Σημάτων}
%βαλε και μια εικονα
\section{Σκοπός και Συνεισφορά της Διπλωματικής Εργασίας}
Στόχος αυτής της διπλωματικής εργασίας είναι η υλοποίηση μιας διεπαφής εγκεφάλου υπολογιστή βασιζόμενη σε μια κατηγορία οπτικών προκλητών δυναμικών που ονομάζονται Steady State Visual Evoked Potentials. Σε αντίθεση όμως με την πλειοψηφία της έρευνας σε αυτόν τον τομέα, όπου γίνεται χρήση ακριβών εγκεφαλογράφων κατασκευασμένων για ιατρική χρήση, σε αυτή την εργασία γίνεται χρήση ενός low-budget εγκεφαλογράφου, πράγμα που αποτελεί και την βασική πρόκληση της εργασίας. Η όλη πειραματική διαδικασία, το hardware για την δημιουργία των οπτικών ερεθισμάτων, η γραφική διεπαφή για την παρακολούθηση των εγκεφαλικών σημάτων σε πραγματικό χρόνο, καθώς και ένα σύνολο συναρτήσεων για την επεξεργασία των εγκεφαλικών σημάτων στην προγραμματιστική γλώσσα Python, σχεδιάστηκαν και υλοποιήθηκαν εξολοκλήρου στα πλαίσια της εργασίας. Επιπλέον, δημιουργήσαμε ένα dataset με όλες τις καταγραφές εγκεφαλογραφήματος που έλαβαν μέρος κατά την διάρκεια της εργασίας, ταξινομημένες ανά άτομο, έτσι ώστε όλα αυτά μαζί να αποτελέσουν εφαλτήριο για περαιτέρω έρευνα πάνω στο συγκεκριμένο θέμα. Όσον αφορά τους αλγορίθμους που χρησιμοποιήθηκαν για την εξαγωγή πληροφορίας από τα εγκεφαλικά σήματα, δοκιμάστηκαν διάφορες μέθοδοι που χρησιμοποιούνται κατά κόρον στην βιβλιογραφία, και είδαμε πως στην offline επεξεργασία των σημάτων, η απόδοση του συστήματος συγκρίνεται με τις state-of-the-art επιδόσεις για low-budget εγκεφαλογράφους, έχοντας όμως πολλά περιθώρια μέχρι να συγκριθεί με τις επιδόσεις πιο ακριβών. Τέλος \textcolor{red}{ΠΕΣ ΤΙ ΘΑ ΚΑΝΕΙΣ ΜΕ ΤΟ REAL TIME}

\section{Διάρθρωση Διπλωματικής Εργασίας}
\par Η εργασία ακολουθεί την εξής πορεία:
\begin{itemize}
	\item Στο κεφάλαιο 2 παρουσιάζεται η σχεδίαση του συνολικού συστήματος και των υπομονάδων του, καθώς και το σύνολο δεδομένων στο οποίο εργαζόμαστε. Γίνεται συσχέτιση με άλλες μεθόδους οι οποίες έχουν σχέση με την εργασία μας, ενώ αφήνουμε τις λεπτομέρειες κάθε υποσυστήματος για να παρουσιαστούν ξεχωριστά.
	\item Στο κεφάλαιο 3 παρουσιάζεται το ζήτημα της αναγνώρισης δράσεων με αξιοποίηση μόνο οπτικής πληροφορίας και εξηγείται η λειτουργία του υποσυστήματος Όρασης Χαμηλού Επιπέδου. Αναλύεται η μέθοδος των Πυκνών Τροχιών \cite{wang_2011} και μια επέκτασή της που λαμβάνει υπόψιν της επιπλέον χαρακτηριστικά πόζας \cite{rohrbach_2012}. Τέλος, περιγράφουμε τη σύνδεση αυτών των εργασιών στη δική μας σχεδιαστική επιλογή.
	\item Στο κεφάλαιο 4 παρουσιάζεται το ζήτημα της ανίχνευσης αντικειμένων και η σύνδεσή του με το σημασιολογικό υποσύστημα. Στη σημασιολογία εντάσσουμε και το υποσύστημα ακουστικής πληροφορίας, οπότε δεν αφιερώνουμε για αυτό ξεχωριστό κεφάλαιο. Γίνεται μια ιστορική επισκόπηση της ανίχνευσης αντικειμένων και προσκηνίου, παρουσιάζεται η μέθοδος των Μοντέλων Παραμορφώσιμων Τμημάτων \cite{felzenszwalb_2008}, των Μοντέλων Μίξης Γκαουσιανών για εξαγωγή προσκηνίου και η Μέθοδος Ταιριάσματος στην ανίχνευση αντικειμένου οπτικά και γλωσσικά. Δικαιολογούμε τις σχεδιαστικές μας επιλογές και παρουσιάζουμε αποτελέσματα ανίχνευσης συνδυάζοντας γλωσσική και οπτική πληροφορία.
	\item Στο κεφάλαιο 5 παρουσιάζεται η διαδικασία εξαγωγής πληροφορίας για τον τύπο λαβής (grasping type) στα frames του βίντεο ως επιπλέον αναγνωριστικό χαρακτηριστικό. Παρουσιάζονται θέματα από την περιοχή των Συνελικτικών Νευρωνικών Δικτύων και η από αρχής υλοποίηση ενός συστήματος ανίχνευσης χεριών.
	\item Στο κεφάλαιο 6 γίνεται ανάλυση των αλγορίθμων και των αποτελεσμάτων ταξινόμησης. Εδώ συγκρίνουμε όλες τις μεθόδους και τα αποτελέσματά μας με άλλες εργασίες.
	\item Στο κεφάλαιο 7 αναπτύσσουμε μια νέα μέθοδο κατάτμησης δράσεων και δείχνουμε τα πειραματικά αποτελέσματα για αυτή πάνω σε βίντεο του συνόλου δεδομένων μας.
	\item Στο κεφάλαιο 8, κλείνουμε αυτή την εργασία παρουσιάζοντας τις συνεισφορές μας και προτείνοντας μελλοντικές ερευνητικές κατευθύνσεις.
\end{itemize}

\end{document}
