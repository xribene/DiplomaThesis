%\chapter{Περίληψη}  % Add the "Abstract" page entry to the Contents
%\addtocontents{toc}{\vspace{1em}}  % Add a gap in the Contents, for aesthetics
 \abstracteng{
 The objective of the current Thesis is the extraction of an algorithm describing a complex human activity performed in an observed video. We start by presenting a generic and abstract methodology for designing a joint video action segmentation and classification system, combining multiple modalities. We further present our implementation of such a system, focusing on fine-grained activities and experimenting on efficiently extracting and combining multiple information channels, from Low-Level Visual information (Dense Trajectories) to sound (subtitles) and semantics (action-object relations and grasping type-action relations), with the last category being suported by text analysis. We extract features using Dense Trajectories, object detection and recognition, both visually, searching inside a dynamic region of interest constructed using a combination of human and foreground detection, and via subtitles and lastly using grasping type information. Tha last type of information is obtained by applying a robust hand detector and then classifying the hand regions using ResNet deep convolutional features. We perform a sequence of experiments regarding feature encoding and classification and reach to an interesting result, that we are able to replace the $\chi^2$ kernel fusion with Tf-Idf encodings or even feature concatenation, slightly increasing classification metrics but especially increase the speed of the classification progress, when a linear SVM is also used. This fact allows this schema to be efficiently used by video segmentation algorithms. Our approach when it comes to video segmentation is minimizing an SVM loss function using probabilities and a novel dynamic programming algorithm, invariant to final segments' length. Our method also returns object handling information in the total extracted algorithm.
 \\\\\\
 \large{\textbf{Keywords}}\\
 Action Segmentation, Video Segmentation, Multiple Modalities, Tf-Idf Encoding
 }
 \clearpage  % Abstract ended, start a new page
