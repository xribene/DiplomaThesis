%\chapter{Abstract}  % Add the "Abstract" page entry to the Contents
%\addtocontents{toc}{\vspace{1em}}  % Add a gap in the Contents, for aesthetics
 %\abstracteng{
 \chapter{Abstract}
\par
 The main purpose of this diploma thesis, is to study whether an inexpensive non-medical graded EEG headset, like Emotiv Epoc, is suitable for implementing a Brain Computer Interface (BCI), based on Steady State Visual Evoked Potentials (SSVEP). First stage of the implementation, is the creation of the device that provides the necessary repeated visual stimuli (RVS). We build, a panel made of four LED arrays, and its driving circuit, which allows the independent selection of stimulus frequency and duty cycle, for every RVS. The next stage, consists of the creation of a application that implements the offline EEG data recording protocol that we designed, which is able to fully control the RVS, and at the same time, providing all the necessary visual and audio cues to the user, in order to follow the protocol. Additionally, we design a Graphical User Interface based in open-source software, for the real-time monitoring of every channel of the EEG signal, in time and frequency domain. During the offline processing, we experiment wιth various methods of feature extraction and classification, as well as some of our modifications of them, and we achieve perfomance up to 90\% accuracy and 20.58 bits/min ITR, results that comfort to the state of the art perfomances of BCIs that use similar materials. Finally, we implement the online part of the BCI, an asynchronous real-time system, where users are asked to navigate their avatar to a desired target, through a virtual maze.
 \\\\\\
 \large{\textbf{Keywords}}\\
Brain Computer Interfaces (BCIs), Steady State Visual Evoked Potentials (SSVEP), Canonical Correlation Analysis (CCA), Self-Paced BCI (asynchronous).
 %}
 \clearpage  % Abstract ended, start a new page
